% Acronyms
\newacronym[description={\glslink{seog}{Search Engine Optimization}.}]
    {seo}{SEO}{Search Engine Optimization}

\newacronym[description={\glslink{json-ldg}{JavaScript Object Notation for Linked Data}.}]
    {json-ld}{JSON-LD}{JavaScript Object Notation for Linked Data}

\newacronym[description={\glslink{serpg}{Search Engine Results Page}.}]
    {serp}{SERP}{Search Engine Results Page}

\newacronym[description={\glslink{csvg}{Comma Separated Values}.}]
    {csv}{CSV}{Comma Separated Values}

\newacronym[description={\glslink{gscg}{Google Search Console}.}]
    {gsc}{GSC}{Google Search Console}

\newacronym[description={\glslink{cmsg}{Content Management System}.}]
    {cms}{CMS}{Content Management System}

\newacronym[description={\glslink{wcagg}{Web Content Accessibility Guidelines}.}]
    {wcag}{WCAG}{Web Content Accessibility Guidelines}

\newacronym[description={\glslink{htmlg}{HyperText Markup Language}.}]
    {html}{HTML}{HyperText Markup Language}

% Glossary entries
\newglossaryentry{seog} {
    name=\glslink{seo}{SEO},
    text=SEO,
    sort=seo,
    description={Con il termine \emph{SEO, Search Engine Optimization} (ing. ottimizzazione per i motori di ricerca) si intende un processo volto a migliorare il posizionamento di un contenuto web nella pagina dei risultati dei motori di ricerca.}
}

\newglossaryentry{on-page-seo} {
    name=\glslink{on-page-seo}{SEO on-page},
    text=on-page,
    sort=on-page-seo,
    description={Con il termine \emph{SEO on-page} si intende il processo di ottimizzazione degli elementi interni a una pagina web, come i metadati, i contenuti, i link, gli URL e altri fattori di ranking.}
}

\newglossaryentry{off-page-seo} {
    name=\glslink{off-page-seo}{SEO off-page},
    text=off-page,
    sort=off-page-seo,
    description={Con il termine \emph{SEO off-page} si intende l'insieme di strategie implementate al di fuori di un sito web, con l'obiettivo di aumentarne l'autorevolezza e la visibilità.}
}

\newglossaryentry{hreflang} {
    name=\glslink{hreflang}{Hreflang},
    text=Hreflang,
    sort=hreflang,
    description={\emph{Hreflang} è un attributo HTML utilizzato per indicare ai motori di ricerca la lingua e il targeting geografico di una pagina web.}
}

\newglossaryentry{tag-canonical} {
    name=\glslink{tag-canonical}{Tag canonical},
    text=tag canonical,
    sort=tag-canonical,
    description={Il \emph{tag canonical} è un elemento HTML utilizzato per evitare problemi relativi a contenuti "duplicati" presenti su più URL.}
}

\newglossaryentry{sitemap} {
    name=\glslink{sitemap}{Sitemap},
    text=sitemap,
    sort=Sitemap,
    description={Una \emph{sitemap} è un file che elenca gerarchicamente le pagine di un sito web.}
}

\newglossaryentry{tag-robots} {
    name=\glslink{tag-robots}{Tag `robots`},
    text=tag `robots`,
    sort=tag-robots,
    description={I \emph{meta tag `robots`} forniscono istruzioni su come scansionare e indicizzare una pagina web.}
}

\newglossaryentry{json-ldg} {
    name=\glslink{json-ld}{JSON-LD},
    text=JSON-LD,
    sort=json-ld,
    description={Con il termine \emph{JSON-LD, JavaScript Object Notation for Linked Data} si intende un formato di dati strutturati che utilizza la sintassi JSON per arricchire il contenuto di una pagina web.}
}

\newglossaryentry{case-sensitive} {
    name=\glslink{case-sensitive}{Case-sensitive},
    text=case-sensitive,
    sort=case-sensitive,
    description={Un'operazione di analisi del testo si definisce \emph{case-sensitive} se distingue tra parole che differiscono solo per l'uso di lettere maiuscole o minuscole.}
}

\newglossaryentry{serpg} {
    name=\glslink{serp}{SERP},
    text=SERP,
    sort=serp,
    description={Con il termine \emph{SERP, Search Engine Results Page} (ing. pagina dei risultati del motore di ricerca) si intende la pagina generata dal motore di ricerca in risposta a una query dell'utente.}
}

\newglossaryentry{csvg} {
    name=\glslink{csv}{CSV},
    text=CSV,
    sort=csv,
    description={Con il termine \emph{CSV, Comma Separated Values} (ing. valori separati da virgola) si intende un formato di file utilizzato per l'importazione ed esportazione di una tabella di dati.}
}

\newglossaryentry{backlink} {
    name=\glslink{backlink}{Backlink},
    text=backlink,
    sort=backlink,
    description={Un \emph{backlink} è un link ipertestuale che punta a un sito web partendo da un altro dominio.}
}

\newglossaryentry{gscg} {
    name=\glslink{gsc}{GSC},
    text=GSC,
    sort=gsc,
    description={Con il termine \emph{GSC, Google Search Console} si intende un servizio offerto da Google per monitorare il posizionamento di un sito web.}
}

\newglossaryentry{wordpress} {
    name=\glslink{wordpress}{WordPress},
    text=WordPress,
    sort=wordpress,
    description={\emph{WordPress} è una piattaforma che consente di creare e gestire siti web (blog, e-commerce, ecc.).}
}

\newglossaryentry{shopify} {
    name=\glslink{shopify}{Shopify},
    text=Shopify,
    sort=shopify,
    description={\emph{Shopify} è una piattaforma che consente di creare e gestire e-commerce.}
}

\newglossaryentry{cmsg} {
    name=\glslink{cms}{CMS},
    text=CMS,
    sort=cms,
    description={Con il termine \emph{CMS, Content Management System} (ing. sistema di gestione dei contenuti) si intende un software, come \gls{wordpress} o \gls{shopify}, che consente di creare e gestire siti web senza richiedere competenze tecniche avanzate.}
}

\newglossaryentry{keyword-difficulty} {
    name=\glslink{keyword-difficulty}{Keyword difficulty},
    text=keyword difficulty,
    sort=keyword-difficulty,
    description={La \emph{keyword difficulty} è una metrica SEO che stima la difficoltà di posizionarsi per una specifica parola chiave.}
}

\newglossaryentry{keyword-stuffing} {
    name=\glslink{keyword-stuffing}{Keyword stuffing},
    text=keyword stuffing,
    sort=keyword-stuffing,
    description={Il \emph{keyword stuffing} è una pratica SEO che consiste nell'utilizzo eccessivo e innaturale di una parola chiave, con l'intento di ottenere un miglior posizionamento.}
}

\newglossaryentry{localhost} {
    name=\glslink{localhost}{localhost},
    text=localhost,
    sort=localhost,
    description={In un ambiente web, \emph{localhost} rappresenta il server locale.}
}

\newglossaryentry{organiche} {
    name=\glslink{organiche}{Ricerche organiche},
    text=organiche,
    sort=ricerche-organiche,
    description={Le \emph{ricerche organiche} sono i risultati non a pagamento generati dai motori di ricerca.}
}

\newglossaryentry{sponsorizzate} {
    name=\glslink{sponsorizzate}{Ricerche sponsorizzate},
    text=sponsorizzate,
    sort=ricerche-sponsorizzate,
    description={Le \emph{ricerche sponsorizzate} sono i risultati a pagamento generati dai motori di ricerca.}
}

\newglossaryentry{long-tail-keywords} {
    name=\glslink{long-tail-keywords}{Long-tail keywords},
    text=long-tail keywords,
    sort=long-tail-keywords,
    description={Le \emph{long-tail keywords} sono frasi con un intento di ricerca preciso.}
}

\newglossaryentry{requisiti} {
    name=\glslink{requisiti}{Requisito software},
    text=requisiti,
    sort=requisiti,
    description={Secondo lo standard IEEE 610.12-1990, un \emph{requisito} è:
    \begin{enumerate}
        \item Una condizione o capacità di cui un utente ha bisogno per risolvere un problema o raggiungere un obiettivo;
        \item Una condizione o capacità che un sistema (o un suo componente) deve soddisfare o possedere per adempiere a un contratto, una norma, una specifica o altri documenti formalmente imposti;
        \item Una rappresentazione documentata di una condizione o capacità come descritto in (1) o (2).
    \end{enumerate}
    }
}

\newglossaryentry{use-case} {
    name=\glslink{use-case}{Caso d'uso},
    text=Casi d'uso,
    sort=caso-duso,
    description={Un \emph{caso d'uso} è una descrizione, testuale e/o visiva, di uno scenario di interazione tra un attore e il sistema. È utilizzato nei processi di ingegneria del software per raccogliere i requisiti funzionali.}
}

\newglossaryentry{user-story} {
    name=\glslink{user-story}{User story},
    text=User story,
    sort=user-story,
    description={Una \emph{user story} è una descrizione informale, in linguaggio naturale, di una funzionalità del sistema, focalizzata sul valore che quest'ultima porta all'utente.}
}

\newglossaryentry{wcagg} {
    name=\glslink{wcag}{WCAG},
    text=WCAG,
    sort=wcag,
    description={Con il termine \emph{WCAG, Web Content Accessibility Guidelines} (ing. linee guida per l'accessibilità dei contenuti web) si intende un insieme di specifiche tecniche finalizzate a rendere i contenuti web più accessibili alle persone con disabilità.}
}

\newglossaryentry{github} {
    name=\glslink{github}{GitHub},
    text=GitHub,
    sort=github,
    description={\emph{GitHub} è un servizio di hosting per progetti software che utilizza \gls{git} per facilitare la collaborazione.}
}

\newglossaryentry{git} {
    name=\glslink{git}{Git},
    text=Git,
    sort=git,
    description={\emph{Git} è un software per il controllo di versione, utilizzato per tracciare le modifiche apportate al codice sorgente durante lo sviluppo.}
}

\newglossaryentry{htmlg} {
    name=\glslink{html}{HTML},
    text=HTML,
    sort=html,
    description={Con il termine \emph{HTML, HyperText Markup Language} (ing. linguaggio di marcatura d'ipertesto) si intende un linguaggio di markup utilizzato per definire la struttura e i contenuti delle pagine web.}
}

