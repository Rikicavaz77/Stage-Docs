% \omiss produces '[...]'
\newcommand{\omissis}{[\dots\negthinspace]}

% Itemize symbols
% see: https://tex.stackexchange.com/a/62497
% \renewcommand{\labelitemi}{$\bullet$}
% \renewcommand{\labelitemii}{$\cdot$}
% \renewcommand{\labelitemiii}{$\diamond$}
% \renewcommand{\labelitemiv}{$\ast$}


\let\Chaptermark\chaptermark
% \Chaptername gives current chapter name
\def\chaptermark#1{\def\Chaptername{#1}\Chaptermark{#1}}
\makeatletter
% \currentname gives the current section name
\newcommand*{\currentname}{\@currentlabelname}
\makeatother

% Uncomment the following line for a different header/footer style
% \pagestyle{fancy} \setlength{\headheight}{14.5pt}
\fancyhead[L]{\fontsize{12}{14.5} \selectfont \thechapter. \Chaptername}
\fancyhead[R]{\fontsize{12}{14.5} \selectfont \currentname}
% Page number always in footer
\cfoot{\thepage}


% Custom hyphenation rules
\hyphenation {
    e-sem-pio
    ex-am-ple
}

% Images path, not using \graphicspath because it doesn't properly work with
% latexmk custom dependencies
\NewCommandCopy{\latexincludegraphics}{\includegraphics}
\renewcommand{\includegraphics}[2][]{\latexincludegraphics[#1]{images/#2}}

% Page format settings
% see: http://wwwcdf.pd.infn.it/AppuntiLinux/a2547.htm
\setlength{\parindent}{14pt}    % first row indentation
\setlength{\parskip}{0pt}       % paragraphs spacing


% Load variables
\newcommand{\myName}{Riccardo Cavalli}
\newcommand{\myID}{2042893}
\newcommand{\myTitle}{Sviluppo di una funzionalità di analisi SEO per l'identificazione delle keywords nelle pagine web}
\newcommand{\myDegree}{Tesi di laurea}
\newcommand{\myUni}{Università degli Studi di Padova}
% For BSc level just use "Corso di Laurea" and don't add "Triennale" to it
\newcommand{\myFaculty}{Corso di Laurea in Informatica}
\newcommand{\myDepartment}{Dipartimento di Matematica ``Tullio Levi-Civita''}
\newcommand{\profTitle}{Prof.ssa}
\newcommand{\myProf}{Ombretta Gaggi}
\newcommand{\myLocation}{Padova}
\newcommand{\myAA}{2024-2025}
\newcommand{\myTime}{Mese AAAA}

% PDF file metadata fields
% when updating them delete the build directory, otherwise they won't change
\begin{filecontents*}{\jobname.xmpdata}
  \Title{Document's title}
  \Author{Author's name}
  \Language{it-IT}
  \Subject{Short description}
  \Keywords{keyword1\sep keyword2\sep keyword3}
\end{filecontents*}


% Acronyms
\newacronym[
  description={\glslink{seog}{Search Engine Optimization}.},
  first={\textit{SEO}\textsuperscript{[g]} (Search Engine Optimization)},
  text=SEO
]{seo}{SEO}{Search Engine Optimization}

\newacronym[
  description={\glslink{jsong}{JavaScript Object Notation}.},
  first={\textit{JSON}\textsuperscript{[g]} (JavaScript Object Notation)},
  text=JSON
]{json}{JSON}{JavaScript Object Notation}

\newacronym[
  description={\glslink{json-ldg}{JavaScript Object Notation for Linked Data}.},
  first={\textit{JSON-LD}\textsuperscript{[g]} (JavaScript Object Notation for Linked Data)},
  text=JSON-LD
]{json-ld}{JSON-LD}{JavaScript Object Notation for Linked Data}

\newacronym[
  description={\glslink{serpg}{Search Engine Results Page}.},
  first={\textit{SERP}\textsuperscript{[g]} (Search Engine Results Page)},
  text=SERP
]{serp}{SERP}{Search Engine Results Page}

\newacronym[
  description={\glslink{csvg}{Comma Separated Values}.},
  first={\textit{CSV}\textsuperscript{[g]} (Comma Separated Values)},
  text=CSV
]{csv}{CSV}{Comma Separated Values}

\newacronym[
  description={\glslink{gscg}{Google Search Console}.},
  first={\textit{GSC}\textsuperscript{[g]} (Google Search Console)},
  text=GSC
]{gsc}{GSC}{Google Search Console}

\newacronym[
  description={\glslink{cmsg}{Content Management System}.},
  first={\textit{CMS}\textsuperscript{[g]} (Content Management System)},
  text=CMS
]{cms}{CMS}{Content Management System}

\newacronym[
  description={\glslink{wcagg}{Web Content Accessibility Guidelines}.},
  first={\textit{WCAG}\textsuperscript{[g]} (Web Content Accessibility Guidelines)},
  text=WCAG
]{wcag}{WCAG}{Web Content Accessibility Guidelines}

\newacronym[
  description={\glslink{htmlg}{HyperText Markup Language}.},
  first={\textit{HTML}\textsuperscript{[g]} (HyperText Markup Language)},
  text=HTML
]{html}{HTML}{HyperText Markup Language}

\newacronym[
  description={\glslink{domg}{Document Object Model}.},
  first={\textit{DOM}\textsuperscript{[g]} (Document Object Model)},
  text=DOM
]{dom}{DOM}{Document Object Model}

\newacronym[
  description={\glslink{susg}{System Usability Scale}.},
  first={\textit{SUS}\textsuperscript{[g]} (System Usability Scale)},
  text=SUS
]{sus}{SUS}{System Usability Scale}

\newacronym[
  description={\glslink{pocg}{Proof of Concept}.},
  first={\textit{PoC}\textsuperscript{[g]} (Proof of Concept)},
  text=PoC
]{poc}{PoC}{Proof of Concept}

\newacronym[
  description={\glslink{yamlg}{Yet Another Markup Language o YAML Ain’t Markup Language}.},
  first={\textit{YAML}\textsuperscript{[g]} (Yet Another Markup Language o YAML Ain’t Markup Language)},
  text=YAML
]{yaml}{YAML}{Yet Another Markup Language o YAML Ain’t Markup Language}

\newacronym[
  description={\glslink{svgg}{Scalable Vector Graphics}.},
  first={\textit{SVG}\textsuperscript{[g]} (Scalable Vector Graphics)},
  text=SVG
]{svg}{SVG}{Scalable Vector Graphics}

\newacronym[
  description={\glslink{cssg}{Cascading Style Sheets}.},
  first={\textit{CSS}\textsuperscript{[g]} (Cascading Style Sheets)},
  text=CSS
]{css}{CSS}{Cascading Style Sheets}


% Glossary entries
\newglossaryentry{seog} {
    name=\glslink{seo}{SEO},
    first={\textit{SEO}\textsuperscript{[g]}},
    text=SEO,
    sort=seo,
    description={Con il termine \textit{SEO, Search Engine Optimization} (in italiano \textit{ottimizzazione per i motori di ricerca}) si intende un processo volto a migliorare il posizionamento di un contenuto web nella pagina dei risultati dei motori di ricerca.}
}

\newglossaryentry{on-page} {
    name=\glslink{on-page}{SEO on-page},
    first={\textit{on-page}\textsuperscript{[g]}},
    text=on-page,
    sort=seo-on-page,
    description={Con il termine \textit{SEO on-page} si intende il processo di ottimizzazione degli elementi interni a una pagina web, come i metadati, i contenuti, i link, gli URL e altri fattori di rankin italiano}
}

\newglossaryentry{off-page} {
    name=\glslink{off-page}{SEO off-page},
    first={\textit{off-page}\textsuperscript{[g]}},
    text=off-page,
    sort=seo-off-page,
    description={Con il termine \textit{SEO off-page} si intende l'insieme di strategie implementate al di fuori di un sito web, con l'obiettivo di aumentarne l'autorevolezza e la visibilità.}
}

\newglossaryentry{hreflang} {
    name=\glslink{hreflang}{Hreflang},
    first={\textit{hreflang}\textsuperscript{[g]}},
    text=hreflang,
    sort=hreflang,
    description={\textit{Hreflang} è un attributo HTML utilizzato per indicare ai motori di ricerca la lingua e il targeting geografico di una pagina web.}
}

\newglossaryentry{tag-canonical} {
    name=\glslink{tag-canonical}{Tag canonical},
    first={\textit{tag canonical}\textsuperscript{[g]}},
    text=tag canonical,
    sort=tag-canonical,
    description={Il \textit{tag canonical} è un elemento HTML utilizzato per evitare problemi relativi a contenuti "duplicati" presenti su più URL.}
}

\newglossaryentry{sitemap} {
    name=\glslink{sitemap}{Sitemap},
    first={\textit{sitemap}\textsuperscript{[g]}},
    text=sitemap,
    sort=sitemap,
    description={Una \textit{sitemap} è un file che elenca gerarchicamente le pagine di un sito web.}
}

\newglossaryentry{tag-robots} {
    name=\glslink{tag-robots}{Tag `robots`},
    first={\textit{tag `robots`}\textsuperscript{[g]}},
    text=tag `robots`,
    sort=tag-robots,
    description={I \textit{meta tag `robots`} forniscono istruzioni su come scansionare e indicizzare una pagina web.}
}

\newglossaryentry{json-ldg} {
    name=\glslink{json-ld}{JSON-LD},
    first={\textit{JSON-LD}\textsuperscript{[g]}},
    text=JSON-LD,
    sort=json-ld,
    description={Con il termine \textit{JSON-LD, JavaScript Object Notation for Linked Data} si intende un formato di dati strutturati che utilizza la sintassi JSON per arricchire il contenuto di una pagina web.}
}

\newglossaryentry{case-sensitive} {
    name=\glslink{case-sensitive}{Case-sensitive},
    first={\textit{case-sensitive}\textsuperscript{[g]}},
    text=case-sensitive,
    sort=case-sensitive,
    description={Un'operazione di analisi del testo si definisce \textit{case-sensitive} se distingue tra parole che differiscono solo per l'uso di lettere maiuscole o minuscole.}
}

\newglossaryentry{serpg} {
    name=\glslink{serp}{SERP},
    first={\textit{SERP}\textsuperscript{[g]}},
    text=SERP,
    sort=serp,
    description={Con il termine \textit{SERP, Search Engine Results Page} (in italiano \textit{pagina dei risultati del motore di ricerca}) si intende la pagina generata dal motore di ricerca in risposta a una query dell'utente.}
}

\newglossaryentry{csvg} {
    name=\glslink{csv}{CSV},
    first={\textit{CSV}\textsuperscript{[g]}},
    text=CSV,
    sort=csv,
    description={Con il termine \textit{CSV, Comma Separated Values} (in italiano \textit{valori separati da virgola}) si intende un formato di file utilizzato per l'importazione ed esportazione di una tabella di dati.}
}

\newglossaryentry{backlink} {
    name=\glslink{backlink}{Backlink},
    first={\textit{backlink}\textsuperscript{[g]}},
    text=backlink,
    sort=backlink,
    description={Un \textit{backlink} è un link ipertestuale che punta a un sito web partendo da un altro dominio.}
}

\newglossaryentry{gscg} {
    name=\glslink{gsc}{GSC},
    first={\textit{GSC}\textsuperscript{[g]}},
    text=GSC,
    sort=gsc,
    description={Con il termine \textit{GSC, Google Search Console} si intende un servizio offerto da Google per monitorare il posizionamento di un sito web.}
}

\newglossaryentry{wordpress} {
    name=\glslink{wordpress}{WordPress},
    first={\textit{WordPress}\textsuperscript{[g]}},
    text=WordPress,
    sort=wordpress,
    description={\textit{WordPress} è una piattaforma che consente di creare e gestire siti web (blog, e-commerce, ecc.).}
}

\newglossaryentry{shopify} {
    name=\glslink{shopify}{Shopify},
    first={\textit{Shopify}\textsuperscript{[g]}},
    text=Shopify,
    sort=shopify,
    description={\textit{Shopify} è una piattaforma che consente di creare e gestire e-commerce.}
}

\newglossaryentry{cmsg} {
    name=\glslink{cms}{CMS},
    first={\textit{CMS}\textsuperscript{[g]}},
    text=CMS,
    sort=cms,
    description={Con il termine \textit{CMS, Content Management System} (in italiano \textit{sistema di gestione dei contenuti}) si intende un software, come \gls{wordpress} o \gls{shopify}, che consente di creare e gestire siti web senza richiedere competenze tecniche avanzate.}
}

\newglossaryentry{keyword-difficulty} {
    name=\glslink{keyword-difficulty}{Keyword difficulty},
    first={\textit{keyword difficulty}\textsuperscript{[g]}},
    text=keyword difficulty,
    sort=keyword-difficulty,
    description={La \textit{keyword difficulty} è una metrica SEO che stima la difficoltà di posizionarsi per una specifica parola chiave.}
}

\newglossaryentry{keyword-stuffing} {
    name=\glslink{keyword-stuffing}{Keyword stuffing},
    first={\textit{keyword stuffing}\textsuperscript{[g]}},
    text=keyword stuffing,
    sort=keyword-stuffing,
    description={Il \textit{keyword stuffing} è una pratica SEO che consiste nell'utilizzo eccessivo e innaturale di una parola chiave, con l'intento di ottenere un miglior posizionamento.}
}

\newglossaryentry{localhost} {
    name=\glslink{localhost}{localhost},
    first={\textit{localhost}\textsuperscript{[g]}},
    text=localhost,
    sort=localhost,
    description={In un ambiente web, \textit{localhost} rappresenta il server locale.}
}

\newglossaryentry{organiche} {
    name=\glslink{organiche}{Ricerche organiche},
    first={\textit{organiche}\textsuperscript{[g]}},
    text=organiche,
    sort=ricerche-organiche,
    description={Le \textit{ricerche organiche} sono i risultati non a pagamento generati dai motori di ricerca.}
}

\newglossaryentry{sponsorizzate} {
    name=\glslink{sponsorizzate}{Ricerche sponsorizzate},
    first={\textit{sponsorizzate}\textsuperscript{[g]}},
    text=sponsorizzate,
    sort=ricerche-sponsorizzate,
    description={Le \textit{ricerche sponsorizzate} sono i risultati a pagamento generati dai motori di ricerca.}
}

\newglossaryentry{long-tail-keywords} {
    name=\glslink{long-tail-keywords}{Long-tail keywords},
    first={\textit{long-tail keywords}\textsuperscript{[g]}},
    text=long-tail keywords,
    sort=long-tail-keywords,
    description={Le \textit{long-tail keywords} sono frasi con un intento di ricerca preciso.}
}

\newglossaryentry{requisiti} {
    name=\glslink{requisiti}{Requisito software},
    first={\textit{requisiti}\textsuperscript{[g]}},
    text=requisiti,
    sort=requisiti,
    description={Secondo lo standard IEEE 610.12-1990, un \textit{requisito} è:
    \begin{enumerate}
        \item Una condizione o capacità di cui un utente ha bisogno per risolvere un problema o raggiungere un obiettivo;
        \item Una condizione o capacità che un sistema (o un suo componente) deve soddisfare o possedere per adempiere a un contratto, una norma, una specifica o altri documenti formalmente imposti;
        \item Una rappresentazione documentata di una condizione o capacità come descritto in (1) o (2).
    \end{enumerate}
    }
}

\newglossaryentry{use-case} {
    name=\glslink{use-case}{Caso d'uso},
    first={\textit{casi d'uso}\textsuperscript{[g]}},
    text=casi d'uso,
    sort=caso-duso,
    description={Un \textit{caso d'uso} è una descrizione, testuale e/o visiva, di uno scenario di interazione tra un attore e il sistema. È utilizzato nei processi di ingegneria del software per raccogliere i requisiti funzionali.}
}

\newglossaryentry{user-story} {
    name=\glslink{user-story}{User story},
    first={\textit{user story}\textsuperscript{[g]}},
    text=user story,
    sort=user-story,
    description={Una \textit{user story} è una descrizione informale, in linguaggio naturale, di una funzionalità del sistema, focalizzata sul valore che quest'ultima porta all'utente.}
}

\newglossaryentry{wcagg} {
    name=\glslink{wcag}{WCAG},
    first={\textit{WCAG}\textsuperscript{[g]}},
    text=WCAG,
    sort=wcag,
    description={Con il termine \textit{WCAG, Web Content Accessibility Guidelines} (in italiano \textit{linee guida per l'accessibilità dei contenuti web}) si intende un insieme di specifiche tecniche finalizzate a rendere i contenuti web più accessibili alle persone con disabilità.}
}

\newglossaryentry{github} {
    name=\glslink{github}{GitHub},
    first={\textit{GitHub}\textsuperscript{[g]}},
    text=GitHub,
    sort=github,
    description={\textit{GitHub} è un servizio di hosting per progetti software che utilizza \gls{git} per facilitare la collaborazione.}
}

\newglossaryentry{git} {
    name=\glslink{git}{Git},
    first={\textit{Git}\textsuperscript{[g]}},
    text=Git,
    sort=git,
    description={\textit{Git} è un software per il controllo di versione, utilizzato per tracciare le modifiche apportate al codice sorgente durante lo sviluppo.}
}

\newglossaryentry{htmlg} {
    name=\glslink{html}{HTML},
    first={\textit{HTML}\textsuperscript{[g]}},
    text=HTML,
    sort=html,
    description={Con il termine \textit{HTML, HyperText Markup Language} (in italiano \textit{linguaggio di marcatura d'ipertesto}) si intende un linguaggio di markup utilizzato per definire la struttura e i contenuti delle pagine web.}
}

\newglossaryentry{domg} {
    name=\glslink{dom}{DOM},
    first={\textit{DOM}\textsuperscript{[g]}},
    text=DOM,
    sort=dom,
    description={Con il termine \textit{DOM, Document Object Model} (in italiano \textit{modello a oggetti del documento}) si intende un modello standard che consente di accedere e aggiornare dinamicamente il contenuto, la struttura e lo stile di un documento.}
}

\newglossaryentry{scala-likert} {
    name=\glslink{scala-likert}{Scala Likert},
    first={\textit{scala Likert}\textsuperscript{[g]}},
    text=scala Likert,
    sort=scala Likert,
    description={Una \textit{scala Likert} è un metodo di valutazione basato su una serie di affermazioni (bilanciate tra positive e negative), a ciascuna delle quali il rispondente assegna un punteggio in base al proprio grado di accordo o disaccordo, secondo un intervallo che generalmente va da “Fortemente d'accordo” a “Fortemente in disaccordo”.}
}

\newglossaryentry{susg} {
    name=\glslink{sus}{SUS},
    first={\textit{SUS}\textsuperscript{[g]}},
    text=SUS,
    sort=sus,
    description={Con il termine \textit{SUS, System Usability Scale} (in italiano \textit{scala di usabilità del sistema}) si intende una \gls{scala-likert} composta da dieci affermazioni (o item), che fornisce una valutazione accurata dell'usabilità di un'applicazione.}
}

\newglossaryentry{jsong} {
    name=\glslink{json}{JSON},
    first={\textit{JSON}\textsuperscript{[g]}},
    text=JSON,
    sort=json,
    description={Con il termine \textit{JSON, JavaScript Object Notation} si intende un formato di testo leggero e facilmente leggibile, comunemente utilizzato per l’archiviazione e lo scambio di dati nelle applicazioni web.}
}

\newglossaryentry{stopword} {
    name=\glslink{stopword}{Stopword},
    first={\textit{stopword}\textsuperscript{[g]}},
    text=stopword,
    sort=stopword,
    description={Le "stopword" sono parole comuni e molto frequenti che vengono escluse o ignorate durante l’analisi del testo.}
}

\newglossaryentry{pocg} {
    name=\glslink{poc}{PoC},
    first={\textit{PoC}\textsuperscript{[g]}},
    text=PoC,
    sort=poc,
    description={Con il termine \textit{PoC, Proof of Concept} (in italiano \textit{prova di fattibilità}) si intende una versione dimostrativa di un progetto, realizzata per verificarne la fattibilità.}
}

\newglossaryentry{yamlg} {
    name=\glslink{yaml}{YAML},
    first={\textit{YAML}\textsuperscript{[g]}},
    text=YAML,
    sort=yaml,
    description={Con il termine \textit{YAML, Yet Another Markup Language o YAML Ain’t Markup Language} (in italiano \textit{un altro linguaggio di markup} o \textit{YAML non è un linguaggio di markup}) si intende uun linguaggio di serializzazione comunemente utilizzato nei file di configurazione.}
}

\newglossaryentry{repository} {
    name=\glslink{repository}{Repository},
    first={\textit{repository}\textsuperscript{[g]}},
    text=repository,
    sort=repository,
    description={Un \textit{repository} (o repo) è un archivio digitale centralizzato che viene comunemente utilizzato per gestire il codice sorgente e altri tipi di risorse durante lo sviluppo software.}
}

\newglossaryentry{javascript} {
    name=\glslink{javascript}{Javascript},
    first={\textit{Javascript}\textsuperscript{[g]}},
    text=Javascript,
    sort=javascript,
    description={\textit{Javascript} è un linguaggio di programmazione utilizzato per definire il comportamento e la logica delle pagine web.}
}

\newglossaryentry{commit} {
    name=\glslink{commit}{Commit},
    first={\textit{commit}\textsuperscript{[g]}},
    text=commit,
    sort=commit,
    description={Un \textit{commit} è un'istantanea (snapshot) dell’intero repository in un determinato momento.}
}

\newglossaryentry{framework} {
    name=\glslink{framework}{Framework},
    first={\textit{framework}\textsuperscript{[g]}},
    text=framework,
    sort=framework,
    description={Un \textit{framework} è una raccolta di componenti, convenzioni e strumenti progettati per supportare e semplificare lo sviluppo software.}
}

\newglossaryentry{continuous integration} {
    name=\glslink{continuous integration}{Continuous integration},
    first={\textit{continuous integration}\textsuperscript{[g]}},
    text=continuous integration,
    sort=continuous integration,
    description={La \textit{continuous integration} è una pratica di sviluppo software che prevede allineamenti frequenti tra l’ambiente locale e quello condiviso. L’integrazione in un repository centralizzato è spesso preceduta dal processo di build e dall’esecuzione dei test automatici, con l’obiettivo di individuare prontamente eventuali errori, migliorare la qualità del software e ridurre il tempo necessario per la convalida delle pull request.}
}

\newglossaryentry{pull request} {
    name=\glslink{pull request}{Pull request},
    first={\textit{pull request}\textsuperscript{[g]}},
    text=pull request,
    sort=pull request,
    description={Una \textit{pull request} è una proposta di integrazione di un insieme di modifiche da un branch a un altro all’interno di un repository.}
}

\newglossaryentry{svgg} {
    name=\glslink{svg}{SVG},
    first={\textit{SVG}\textsuperscript{[g]}},
    text=SVG,
    sort=svg,
    description={Con il termine \textit{SVG, Scalable Vector Graphics} (in italiano \textit{grafica vettoriale scalabile}) si intende un formato di file vettoriale spesso utilizzato per visualizzare elementi grafici nelle pagine web, in quanto può essere ridimensionato senza perdere risoluzione.}
}

\newglossaryentry{cssg} {
    name=\glslink{css}{CSS},
    first={\textit{CSS}\textsuperscript{[g]}},
    text=CSS,
    sort=css,
    description={Con il termine \textit{CSS, Cascading Style Sheets} (in italiano \textit{fogli di stile a cascata}) si intende un linguaggio utilizzato per definire lo stile e la formattazione delle pagine web.}
}

\renewcommand*{\glspostdescription}{}
\makeglossaries

\bibliography{appendix/bibliography}

\defbibheading{bibliography} {
    \cleardoublepage
    \phantomsection
    \addcontentsline{toc}{chapter}{\bibname}
    \chapter*{\bibname\markboth{\bibname}{\bibname}}
}

% Spacing between entries
\setlength\bibitemsep{1.5\itemsep}

\DeclareBibliographyCategory{opere}
\DeclareBibliographyCategory{web}

\addtocategory{opere}{womak:lean-thinking}
\addtocategory{web}{site:agile-manifesto}

\defbibheading{opere}{\section*{Riferimenti bibliografici}}
\defbibheading{web}{\section*{Siti Web consultati}}


\captionsetup{
    tableposition=top,
    figureposition=bottom,
    font=small,
    format=hang,
    labelfont=bf
}

\hypersetup{
    %hyperfootnotes=false,
    %pdfpagelabels,
    colorlinks=true,
    linktocpage=true,
    pdfstartpage=1,
    pdfstartview=,
    breaklinks=true,
    pdfpagemode=UseNone,
    pageanchor=true,
    pdfpagemode=UseOutlines,
    plainpages=false,
    bookmarksnumbered,
    bookmarksopen=true,
    bookmarksopenlevel=1,
    hypertexnames=true,
    pdfhighlight=/O,
    %nesting=true,
    %frenchlinks,
    urlcolor=webbrown,
    linkcolor=RoyalBlue,
    citecolor=webgreen
    %pagecolor=RoyalBlue,
}

% Delete all links and show them in black
\if \isprintable 1
    \hypersetup{draft}
\fi

% Listings setup
\lstset{
    language=[LaTeX]Tex,%C++,
    keywordstyle=\color{RoyalBlue}, %\bfseries,
    basicstyle=\small\ttfamily,
    %identifierstyle=\color{NavyBlue},
    commentstyle=\color{Green}\ttfamily,
    stringstyle=\rmfamily,
    numbers=none, %left,%
    numberstyle=\scriptsize, %\tiny
    stepnumber=5,
    numbersep=8pt,
    showstringspaces=false,
    breaklines=true,
    frameround=ftff,
    frame=single
}

\definecolor{webgreen}{rgb}{0,.5,0}
\definecolor{webbrown}{rgb}{.6,0,0}

\newcommand{\sectionname}{sezione}
\addto\captionsitalian{\renewcommand{\figurename}{Figura}
                       \renewcommand{\tablename}{Tabella}}

\newcommand{\glsfirstoccur}{\ap{{[g]}}}

\newcommand{\intro}[1]{\emph{\textsf{#1}}}

% Risks environment
\newcounter{riskcounter}                % define a counter
\setcounter{riskcounter}{0}             % set the counter to some initial value

%%%% Parameters
% #1: Title
\newenvironment{risk}[1]{
    \refstepcounter{riskcounter}        % increment counter
    \par \noindent                      % start new paragraph
    \textbf{\arabic{riskcounter}. #1}   % display the title before the content of the environment is displayed
}{
    \par\medskip
}

\newcommand{\riskname}{Rischio}

\newcommand{\riskdescription}[1]{\textbf{\\Descrizione:} #1.}

\newcommand{\risksolution}[1]{\textbf{\\Soluzione:} #1.}

% Use case environment
\newcounter{usecasecounter}             % define a counter
\setcounter{usecasecounter}{0}          % set the counter to some initial value

%%%% Parameters
% #1: ID
% #2: Nome
\newenvironment{usecase}[2]{
    \renewcommand{\theusecasecounter}{\usecasename #1}  % this is where the display of
                                                        % the counter is overwritten/modified
    \refstepcounter{usecasecounter}             % increment counter
    %\vspace{10pt}
    \section*{\large \usecasename #1: #2}   % display the title before the
                                            % content of the environment is displayed
}{
    %\medskip
}

\newcommand{\usecasename}{UC}

\newcommand{\usecaseactors}[1]{\par\noindent\textbf{Attori Principali}: #1\medskip}
\newcommand{\usecasepre}[1]{\par\noindent\textbf{Precondizioni}: #1\medskip}
\newcommand{\usecasedesc}[1]{\par\noindent\textbf{Descrizione}: #1\medskip}
\newcommand{\usecasepost}[1]{\par\noindent\textbf{Postcondizioni}: #1\medskip}
\newcommand{\usecasealt}[1]{\par\noindent\textbf{Scenario Alternativo}: #1\medskip}
\newcommand{\usecasesub}[1]{\par\noindent\textbf{Sottocasi d'uso}: #1\medskip}
\newcommand{\usecaseinc}[1]{\par\noindent\textbf{Inclusioni}: #1\medskip}
\newcommand{\usecaseext}[1]{\par\noindent\textbf{Estensioni}: #1\medskip}
\newcommand{\usecaseactorsEnv}[1]{\par\noindent\textbf{Attori Principali}: #1}
\newcommand{\usecasepreEnv}[1]{\par\noindent\textbf{Precondizioni}: #1}
\newcommand{\usecasedescEnv}[1]{\par\noindent\textbf{Descrizione}: #1}
\newcommand{\usecasepostEnv}[1]{\par\noindent\textbf{Postcondizioni}: #1}
\newcommand{\usecasealtEnv}[1]{\par\noindent\textbf{Scenario Alternativo}: #1}
\newcommand{\usecasesubEnv}[1]{\par\noindent\textbf{Sottocasi d'uso}: #1}
\newcommand{\usecaseincEnv}[1]{\par\noindent\textbf{Inclusioni}: #1}
\newcommand{\usecaseextEnv}[1]{\par\noindent\textbf{Estensioni}: #1}

% Namespace description environment
\newenvironment{namespacedesc}{
    \vspace{10pt}
    \par \noindent  % start new paragraph
    \begin{description}
}{
    \end{description}
    \medskip
}

\newcommand{\classdesc}[2]{\item[\textbf{#1:}] #2}
