\cleardoublepage
\phantomsection
\pdfbookmark{Sommario}{Sommario}
\begingroup
\let\clearpage\relax
\let\cleardoublepage\relax
\let\cleardoublepage\relax

\chapter*{Sommario}

Il presente documento illustra il lavoro svolto dal laureando Cavalli Riccardo durante il periodo di stage interno presso l’Università degli Studi di Padova. Il tirocinio, della durata di circa 320 ore, è stato assegnato dalla Prof.ssa Ombretta Gaggi, con la collaborazione del Prof. Claudio Palazzi in qualità di tutor interno. Lo stage si è svolto nel periodo compreso tra il 7 aprile e il 9 giugno 2025.

\vspace{10pt}
\noindent Il progetto prevede lo sviluppo di uno strumento SEO per l’identificazione e l’analisi delle parole chiave all’interno di una pagina web. Le parole chiave possono essere estratte dal meta tag keywords (se presente), individuate automaticamente dal sistema in base a determinati criteri (come la frequenza), oppure inserite manualmente dall’utente. In aggiunta all’analisi delle parole chiave, questo strumento - da integrare in un’estensione per browser - deve consentire l’evidenziazione di tutte le occorrenze di una parola chiave nella pagina. La tesi descrive le fasi di sviluppo del progetto, seguendo i principi dell’ingegneria del software.

\vspace{10pt}
\noindent Il periodo di stage è stato suddiviso in tre fasi. La prima è stata dedicata allo studio delle soluzioni presenti sul mercato e alla stesura di una relazione sulle funzionalità, i vantaggi e gli svantaggi di ciascuno strumento, fornendo così una base per la formalizzazione dei requisiti. La seconda fase è stata riservata allo sviluppo delle funzionalità di analisi SEO. Infine, la terza fase è stata dedicata al collaudo del software.

%\vfill

%\selectlanguage{english}
%\pdfbookmark{Abstract}{Abstract}
%\chapter*{Abstract}

%\selectlanguage{italian}

\endgroup

\vfill
