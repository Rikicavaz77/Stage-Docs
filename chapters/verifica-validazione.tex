\chapter{Verifica e validazione}
\label{cap:verifica-validazione}

\par In questa sezione sono riportati i test effettuati durante lo svolgimento del progetto. La fase di testing è stata suddivisa in due sotto-fasi: verifica e validazione (V\&V). La verifica consiste nell’accertare che il software sia conforme alle specifiche, e può essere condotta senza eseguire il codice. Le attività di analisi statica includono la revisione e l’ispezione dei processi, dei requisiti, del design e del codice sorgente. Quest’ultima può avvenire tramite una lettura approfondita del codice o mediante una revisione sistematica e mirata. La validazione, invece, riguarda l’analisi dinamica del software, finalizzata ad accertare il soddisfacimento dei requisiti e delle aspettative dal punto di vista dell’utente finale. Le attività di validazione comprendono sia test automatici che manuali.

\section{Test automatici}