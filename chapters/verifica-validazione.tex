\chapter{Verifica e validazione}
\label{cap:verifica-validazione}

\intro{In questa sezione viene discussa la fase di testing, che include test manuali e automatici, fondamentali per valutare la qualità del software e il raggiungimento degli obiettivi stabiliti.}

\section{Introduzione}
\label{sec:introduzione-testing}

\par In questa sezione sono riportati i test effettuati durante lo svolgimento del progetto. La fase di testing è stata suddivisa in due sotto-fasi: verifica e validazione (V\&V). La verifica consiste nell’accertare che il software sia conforme alle specifiche, e può essere condotta senza eseguire il codice. Le attività di analisi statica includono la revisione e l’ispezione dei processi, dei requisiti, del design e del codice sorgente. Quest’ultima può avvenire tramite una lettura approfondita del codice o mediante una revisione sistematica e mirata. La validazione, invece, riguarda l’analisi dinamica del software, finalizzata ad accertare il soddisfacimento dei requisiti e delle aspettative dal punto di vista dell’utente finale. Le attività di validazione comprendono sia test automatici che manuali.

\lstdefinelanguage{JavaScript}{
  keywords={break, case, catch, continue, debugger, default, delete, do, else, finally, for, function, if, in, instanceof, new, return, switch, this, throw, try, typeof, var, void, while, with, let, const},
  keywordstyle=\color{blue}\bfseries,
  ndkeywords={class, export, boolean, throw, implements, import, this},
  ndkeywordstyle=\color{darkgray}\bfseries,
  identifierstyle=\color{black},
  sensitive=false,
  comment=[l]{//},
  morecomment=[s]{/*}{*/},
  commentstyle=\color{gray}\ttfamily,
  stringstyle=\color{red}\ttfamily,
  morestring=[b]',
  morestring=[b]"
}

\section{Test automatici}

\par Il testing automatizzato è una tecnica che prevede la configurazione di un ambiente di test e la progettazione di una o più suite che vengono eseguite automaticamente, confrontando i risultati effettivi con quelli attesi. Ogni test dovrebbe essere veloce, indipendente, affidabile e riutilizzabile. I test automatici non sostituiscono i test manuali né eliminano la necessità di una revisione “umana”; tuttavia, forniscono una misura oggettiva della qualità del software e contribuiscono a ottimizzare il ciclo di sviluppo, individuando rapidamente errori che potrebbero sfuggire all’osservazione manuale, specialmente in contesti di integrazione continua.

\vspace{10pt}
\par\noindent La progettazione, scrittura e manutenzione dei test automatici hanno richiesto uno sforzo considerevole; tuttavia, data la ridotta finestra temporale dello stage, l’automatizzazione si è rivelata essenziale nel lungo periodo. Essa ha infatti fornito un riscontro continuo sulla qualità del software durante lo sviluppo, riducendo il carico di lavoro legato all’esecuzione dei test manuali. Inoltre, ha consentito di raggiungere una copertura del codice del 100\%, risultato difficilmente ottenibile con il solo testing manuale.

\begin{figure}[H]
  \centering 
  \fbox{\includegraphics[width=0.9\columnwidth]{test/coverage.png}}
  \caption{Copertura del codice - report di Codecov}
\end{figure}

\par\noindent I test automatici vengono eseguiti a ogni apertura, aggiornamento o chiusura di una pull request, garantendo che tutto il codice rilasciato superi i test e mantenga una copertura uniforme. Dal punto di vista architetturale e organizzativo, i test rispecchiano fedelmente la struttura del core dell’estensione, risultando più leggibili e manutenibili. 

\vspace{10pt}
\begin{samepage}
  \dirtree{%
    .1 tests.
    .2 controller.
    .2 model.
    .2 services.
    .3 strategy.
    .2 utils.
    .2 view.
  }
\end{samepage}

\vspace{10pt}
\par\noindent Per mantenere l’isolamento tra l’ambiente di test e il resto dell’applicazione, ciascun file testato deve terminare con la seguente porzione di codice:

\vspace{10pt}
\begin{samepage}
\begin{lstlisting}[language=JavaScript]
  /* istanbul ignore next */
  if (typeof module !== 'undefined' && typeof module.exports !== 'undefined') {
    module.exports = KeywordHighlighter;
  }
\end{lstlisting}
\end{samepage}

\vspace{10pt}
\par\noindent Questo approccio consente di importare i moduli all’interno dei file di test tramite la seguente istruzione:

\vspace{10pt}
\begin{samepage}
\begin{lstlisting}[language=JavaScript]
  const KeywordHighlighter = require('@keyword/services/keyword_highlighter');
\end{lstlisting}
\end{samepage}

\vspace{10pt}
\par\noindent Di seguito sono riportati i test di unità e di integrazione.

\renewcommand{\arraystretch}{1.5}
\begin{longtable}{>{\raggedright\arraybackslash}p{0.65\textwidth} >{\raggedright\arraybackslash}p{0.25\textwidth}}
\caption{Tabella dei test automatici}
\label{tab:test-automatici} \\
\hline\hline
\textbf{Suite di test} & \textbf{\% di superamento dei test}\\
\endfirsthead
    
\caption[]{Tabella dei test automatici (continua)} \\
\hline\hline
\textbf{Suite di test} & \textbf{\% di superamento dei test} \\ 
\endhead
    
\multicolumn{2}{r}{{Continua nella prossima pagina}} \\ 
\endfoot
    
\hline
\endlastfoot

\hline
\textbf{Keyword} \textit{(keyword.test.js)}: verifica il corretto funzionamento del modello relativo alle parole chiave & 100\% \\
\hline
\textbf{AnalysisResultView} \textit{(analysis\_result\_view.test.js)}: verifica il corretto funzionamento del componente responsabile della visualizzazione dei risultati relativi all’analisi di una parola chiave & 100\% \\
\hline 
\textbf{KeywordListView} \textit{(keyword\_list\_view.test.js)}: verifica il corretto funzionamento del componente responsabile della visualizzazione delle liste di parole chiave per ciascuna tipologia & 100\% \\
\hline 
\textbf{KeywordView} \textit{(view.test.js)}: verifica il corretto funzionamento della view principale, responsabile della gestione della dashboard e delle sotto-view & 100\% \\
\hline 
\textbf{KeywordController} \textit{(controller.test.js)}: verifica il corretto funzionamento delle operazioni di gestione delle parole chiave (funzionalità non dipendenti dal DOM e non sufficientemente specifiche da giustificare una suite dedicata) & 100\% \\
\hline 
\textbf{KeywordController - DOM} \textit{(controller\_dom.test.js)}: verifica il corretto funzionamento delle operazioni del controller che richiedono l’interazione con il \gls{dom} & 100\% \\
\hline 
\textbf{KeywordController - sorting} \textit{(controller\_sorting.test.js)}: verifica le funzionalità di ordinamento delle parole chiave & 100\% \\
\hline
\textbf{KeywordController - filtering} \textit{(controller\_filtering.test.js)}: verifica le funzionalità di filtraggio delle parole chiave & 100\% \\
\hline
\textbf{KeywordController - pagination} \textit{(controller\_pagination.test.js)}: verifica la logica di gestione della paginazione delle parole chiave & 100\% \\
\hline
\textbf{KeywordController - highlight} \textit{(controller\_highlight.test.js)}: verifica la logica di gestione dell’evidenziazione delle parole chiave nella pagina & 100\% \\
\hline
\textbf{KeywordController - events} \textit{(controller\_events.test.js)}: verifica il corretto funzionamento dell’associazione (binding) degli eventi & 100\% \\
\hline
\textbf{KeywordController - init} \textit{(controller\_init.test.js)}: verifica le funzionalità di inizializzazione e aggiornamento del controller, nonché l’integrazione tra i componenti dell’architettura & 100\% \\
\hline
\textbf{TreeWalkerManager} \textit{(tree\_walker\_manager.test.js)}: verifica le funzionalità di gestione dell’oggetto TreeWalker & 100\% \\
\hline
\textbf{TextProcessor} \textit{(text\_processor.test.js)}: verifica il corretto comportamento delle funzioni di utilità legate all’analisi e all’elaborazione testuale & 100\% \\
\hline
\textbf{TagAccessor} \textit{(tag\_accessor.test.js)}: verifica il corretto comportamento delle funzioni di utilità dedicate all’accesso e alla lettura del contenuto dei tag \gls{html} & 100\% \\
\hline
\textbf{WordCounter} \textit{(word\_counter.test.js)}: verifica le funzionalità di conteggio delle parole nella pagina e di estrazione di quelle più frequenti & 100\% \\
\hline
\textbf{KeywordHighlighter} \textit{(keyword\_highlighter.test.js)}: verifica le funzionalità di evidenziazione delle parole chiave all’interno della pagina & 100\% \\
\hline
\textbf{KeywordAnalyzer} \textit{(keyword\_analyzer.test.js)}: verifica le funzionalità dedicate all’analisi delle parole chiave & 100\% \\
\hline
\textbf{KeywordAnalysisStrategy} \textit{(keyword\_analysis\_strategy.test.js)}: verifica il corretto comportamento della struttura astratta, assicurandosi che l’istanziazione diretta sia impedita e che i metodi non implementati generino un errore & 100\% \\
\hline
\textbf{StagedAnalysisStrategy} \textit{(staged\_analysis\_strategy.test.js)}: verifica il corretto funzionamento dell’implementazione concreta della strategy, basata su un’analisi “per fasi” (staged) & 100\% \\
\hline
\textbf{AllInOneAnalysisStrategy} \textit{(all\_in\_one\_analysis\_strategy.test.js)}: verifica il corretto funzionamento dell’implementazione concreta della strategy, basata su un’analisi compatta e unificata & 100\% \\
\hline
\textbf{Utils} \textit{(utils.test.js)}: verifica il corretto comportamento delle funzioni di utilità & 100\% \\
\end{longtable}

\section{Test manuali}

\par L’estensione è stata testata su un insieme di pagine web differenti per dimensione (in termini di profondità del DOM) e lingua (inglese o italiana), con l’obiettivo di verificarne il corretto funzionamento mediante osservazione manuale.

\renewcommand{\arraystretch}{1.5}
\begin{tabularx}{\textwidth}{>{\raggedright\arraybackslash}X >{\raggedright\arraybackslash}X}
\caption{Tabella dei test manuali}
\label{tab:test-manuali} \\
\hline\hline
\textbf{Sito web} & \textbf{Esito dei test}\\
\endfirsthead
    
\caption[]{Tabella dei test manuali (continua)} \\
\hline\hline
\textbf{Sito web} & \textbf{Esito dei test} \\ 
\endhead
    
\multicolumn{2}{r}{{Continua nella prossima pagina}} \\ 
\endfoot
    
\hline
\endlastfoot

\hline
\textbf{W3Schools} - JavaScript Tutorial (\url{https://www.w3schools.com/js/}) & Tutti i test manuali sono stati superati con successo, inclusa l'analisi di parole chiave “spezzate” su più tag \gls{html} (es. “JavaScript Tutorial”). L’unica eccezione riscontrata nelle prime fasi di sviluppo riguardava il mancato rilevamento dei meta tag \textit{keywords} e \textit{description}. Il problema era dovuto al fatto che, contrariamente alle convenzioni più comuni, il valore dell’attributo \textit{name} iniziava con una lettera maiuscola. La soluzione adottata è stata l’aggiunta del flag di confronto “case-insensitive” nel selettore \\
\hline
\textbf{W3Schools} - JavaScript Regular Expressions (\url{https://www.w3schools.com/js/js\_regexp.asp}) & Tutti i test manuali sono stati superati con successo \\
\hline
\textbf{Corriere della Sera} - HomePage (\url{https://www.corriere.it/}) & Tutti i test manuali sono stati superati con successo, sebbene con prestazioni non sempre ottimali, a causa della complessità del sito in questione \\
\hline
Siti web partecipanti al concorso \textbf{Accattivante Accessibile}, organizzato dall’Università degli Studi di Padova (\url{https://web.math.unipd.it/CAA/classifica.html}) & Tutti i test manuali sono stati superati con successo. Durante l’analisi del sito \textit{BookOverflow}, è emerso un caso particolare non gestito: una singola parola “spezzata” su più tag HTML (es. “BookOverflow”). Questa situazione è stata valutata come un’eccezione poco significativa, non tale da giustificare una gestione più granulare, che avrebbe inciso negativamente sulle prestazioni \\
\hline
\textbf{Nasce, Cresce, Ignora} - The Killer, la recensione (\url{https://nascecresceignora.it/the-killer-la-recensione-un-thriller-freddo-e-glaciale/}) & Tutti i test manuali sono stati superati con successo, inclusa l'analisi di parole chiave “spezzate” su più tag HTML (es. “David Fincher”) e l’estrazione delle keyphrase più frequenti (es. “The Killer”) \\
\hline
\textbf{Nasce, Cresce, Ignora} - Pixar: I migliori film (\url{https://nascecresceignora.it/pixar-i-migliori-film-secondo-letterboxd/}) & Tutti i test manuali sono stati superati con successo \\
\hline
\textbf{Nasce, Cresce, Ignora} - Death Stranding 2: On The Beach (\url{https://nascecresceignora.it/death-stranding-2-on-the-beach-pubblicato-nuovo-trailer-gameplay/}) & Tutti i test manuali sono stati superati con successo \\
\hline
\textbf{Nasce, Cresce, Ignora} - The Last of Us in concerto (\url{https://nascecresceignora.it/the-last-of-us-in-concerto-gustavo-santaolalla-annuncia-2-date-italia/}) & Tutti i test manuali sono stati superati con successo \\
\hline
\textbf{la Repubblica} - HomePage (\url{https://www.repubblica.it/}) & Tutti i test manuali sono stati superati con successo \\
\hline
\textbf{TradingView} - HomePage (\url{https://it.tradingview.com/}) & Tutti i test manuali sono stati superati con successo, sebbene con prestazioni non sempre ottimali, a causa della complessità del sito in questione \\
\hline
\textbf{GitHub} - HomePage (\url{https://github.com/}) & Tutti i test manuali sono stati superati con successo \\
\hline
\textbf{STEM Unipd} - HomePage (\url{https://stem.elearning.unipd.it/}) & Tutti i test manuali sono stati superati con successo. Data la presenza di numerosi tag nascosti, la maggior parte delle parole chiave più frequenti non è direttamente visibile \\
\end{tabularx}
