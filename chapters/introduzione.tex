\chapter{Introduzione}
\label{cap:introduzione}

\par Uno degli obiettivi principali di chi pubblica contenuti sul web è garantirne la massima visibilità. In altri termini, sviluppatori, redattori e altre figure professionali operanti nel settore digitale mirano a ottenere un buon posizionamento nella \gls{serp}, affinché i propri contenuti possano raggiungere il pubblico più ampio possibile.

\par\noindent Esempio di utilizzo di un termine nel glossario
\gls{htmlg}.

\par\noindent Esempio di citazione in linea
\cite{site:semver}.

\par\noindent Esempio di citazione nel pie' di pagina 
%citazione\footcite{womak:lean-thinking}

\section{L'azienda}

\par Descrizione dell'azienda.

\section{L'idea}

\par Introduzione all'idea dello stage.

\section{Organizzazione del testo}

\begin{description}
    \item[{\hyperref[cap:descrizione-progetto]{Il secondo capitolo}}] descrive ...
    
    %\item[{\hyperref[cap:processi-metodologie]{Il terzo capitolo}}] approfondisce ...
    
    \item[{\hyperref[cap:analisi-requisiti]{Il quarto capitolo}}] approfondisce ...
    
    \item[{\hyperref[cap:progettazione-codifica]{Il quinto capitolo}}] approfondisce ...
    
    \item[{\hyperref[cap:verifica-validazione]{Il sesto capitolo}}] approfondisce ...
    
    %\item[{\hyperref[cap:conclusioni]{Nel settimo capitolo}}] descrive ...
\end{description}

\par Riguardo la stesura del testo, relativamente al documento sono state adottate le seguenti convenzioni tipografiche:
\begin{itemize}
	\item gli acronimi, le abbreviazioni e i termini ambigui o di uso non comune menzionati vengono definiti nel glossario, situato alla fine del presente documento;
	\item per la prima occorrenza dei termini riportati nel glossario viene utilizzata la seguente nomenclatura: \emph{parola}\glsfirstoccur;
	\item i termini in lingua straniera o facenti parti del gergo tecnico sono evidenziati con il carattere \emph{corsivo}.
\end{itemize}
