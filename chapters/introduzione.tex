\chapter{Introduzione}
\label{cap:introduzione}

\par Introduzione al contesto applicativo.

\section{La Proponente}

\par Descrizione dell'azienda.

\section{L'idea}

\par Introduzione all'idea dello stage.

\section{Glossario}

\par Allo scopo di evitare incomprensioni legate al linguaggio utilizzato, è stato inserito, in calce al documento, un glossario in cui ogni termine è corredato da una spiegazione volta a illustrarne il significato. Oltre a termini ambigui o poco comuni, il glossario include anche acronimi e abbreviazioni. La prima occorrenza di ciascun termine è evidenziata con la seguente formattazione: \emph{termine}\glsfirstoccur. Di seguito sono riportati due esempi di visualizzazione dei termini inclusi nel glossario:

\begin{itemize}
    \item \gls{html};
    \item \gls{github}.
\end{itemize}

\section{Convenzioni tipografiche}

\par Riguardo la stesura del testo, relativamente al documento sono state adottate le seguenti convenzioni tipografiche:
\begin{itemize}
	\item i termini in lingua straniera o facenti parti del gergo tecnico sono evidenziati con il carattere \emph{corsivo}.
\end{itemize}
