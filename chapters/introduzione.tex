\chapter{Introduzione}
\label{cap:introduzione}

\par Uno degli obiettivi principali di chi pubblica contenuti sul web è garantirne la massima visibilità. In altri termini, sviluppatori, redattori e altre figure professionali operanti nel settore digitale mirano a ottenere un buon posizionamento nella \gls{serp}, affinché i propri contenuti possano raggiungere il pubblico più ampio possibile. In un panorama in cui gli ambiti e le aree di maggiore interesse per l’utenza sono già ampiamente coperti, realizzare un sito accattivante e ricco di funzionalità non è più sufficiente. Ritagliarsi uno spazio in un mercato saturo è ancora possibile, ma comporta uno sforzo significativo in termini di risorse e competenze, legato principalmente al processo di ottimizzazione \gls{seo}.

\vspace{10pt}
\par\noindent Per “ottimizzazione SEO” si intende l’insieme di strategie e tecniche finalizzate ad aumentare la visibilità di una pagina web. I motori di ricerca, in risposta a una \textit{query} dell’utente, generano un elenco di risultati organici (non a pagamento), che include collegamenti alle pagine ritenute più pertinenti in base a molteplici fattori. L’obiettivo dell’ottimizzazione \gls{seo} è migliorare il posizionamento di una pagina all’interno di questo elenco, incrementando così le probabilità che venga visitata dagli utenti.

\vspace{10pt}
\par\noindent I risultati restituiti dai motori di ricerca sono strutturati secondo un sistema di paginazione. Numerosi studi evidenziano come gli utenti attribuiscano maggiore credibilità ai siti collocati nelle primissime posizioni della \gls{serp}, ignorando quasi del tutto quelli che compaiono nelle pagine successive. Le strategie \gls{seo} si suddividono in due macro-aree: \gls{on-page} e \gls{off-page}. Tra i fattori \gls{on-page} che influenzano il \textit{ranking}, rivestono particolare importanza la scelta delle parole chiave e il loro utilizzo in punti strategici della pagina.

\section{La Proponente}

\par Il progetto di stage, svolto internamente all’Università degli Studi di Padova, è stato commissionato dalla Prof.ssa Ombretta Gaggi, che ha ricoperto il ruolo di tutor esterno. Si tratta di un progetto accademico continuativo, avviato nel 2024 nell’ambito di un tirocinio, finalizzato allo sviluppo di un’estensione web orientata all’accessibilità e all’ottimizzazione \gls{seo}.

\section{L'idea}

\par Ai fini dell’ottimizzazione \gls{seo}, l’inserimento delle parole chiave in sezioni strategiche della pagina risulta fondamentale. Con il termine “sezioni strategiche” si intendono elementi \gls{html} particolarmente rilevanti in ottica \gls{seo}, tra cui:
\begin{itemize}
    \item Il tag title;
    \item La meta description;
    \item I tag di intestazione (heading);
    \item Il testo alternativo delle immagini.
\end{itemize}

\vspace{5pt}
\par\noindent Una volta selezionate le parole chiave, verificarne la presenza nelle aree strategiche e analizzarne la distribuzione all’interno del \gls{dom} sono operazioni complesse, ripetitive e dispendiose se svolte manualmente. Da qui nasce l’opportunità - approfondita nel \hyperref[cap:descrizione-progetto]{capitolo successivo} - di sviluppare un’estensione web che automatizzi tale processo.

\section{Glossario}

\par Allo scopo di evitare incomprensioni legate al linguaggio utilizzato, è stato inserito, in calce al documento, un glossario in cui ogni termine è corredato da una spiegazione volta a illustrarne il significato. Oltre a termini ambigui o poco comuni, il glossario include anche acronimi e abbreviazioni. La prima occorrenza di ciascun termine è evidenziata con la seguente formattazione: \textit{termine}\glsfirstoccur.

\section{Convenzioni tipografiche}

\par Relativamente alla stesura del testo, sono state adottate le seguenti convenzioni tipografiche:
\begin{itemize}
	\item Il \textit{corsivo} è utilizzato per termini tecnici, espressioni in lingua straniera e voci del glossario alla loro prima occorrenza;
	\item Il \textbf{grassetto} è riservato esclusivamente alle intestazioni o ai concetti su cui si intende porre una forte enfasi;
	\item Nel corpo del testo vengono utilizzate unicamente le virgolette alte doppie (“...”).
\end{itemize}
