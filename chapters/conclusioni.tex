\chapter{Conclusioni}
\label{cap:conclusioni}

\section{Consuntivo finale}

\par Il periodo di stage si è svolto dal 7 aprile al 9 giugno 2025, per un totale di 9 settimane. Rispetto al piano di lavoro standard è stata aggiunta una settimana, interamente dedicata all’analisi delle soluzioni esistenti e alla stesura di una relazione. Nelle prime sei settimane sono state svolte 40 ore di lavoro ciascuna, mentre nelle ultime tre l’impegno è stato di 30 ore settimanali, per un totale complessivo di 330 ore.

\section{Raggiungimento degli obiettivi}

\par Come illustrato nelle tabelle \ref{tab:requisiti-implementazione} e \ref{tab:test-automatici}, tutti i requisiti funzionali sono stati soddisfatti, con una copertura del codice e un superamento dei test pari al 100\%. Nella tabella \ref{tab:requisiti-implementazione} sono riportate le classi che hanno contribuito al soddisfacimento di ciascun requisito. Tra i requisiti di vincolo, di dominio e di qualità, l’unico obiettivo non raggiunto riguarda l’attivazione automatica dell’estensione su tutte le tab del browser. Questo requisito, previsto nel piano di lavoro e per il quale era stata già progettata l’interfaccia grafica, non è stato implementato per dedicare maggiori risorse al collaudo e all’ottimizzazione delle funzionalità preesistenti.

\section{Conoscenze acquisite}

\par Durante il periodo di stage ho avuto l’opportunità di approfondire i componenti fondamentali per lo sviluppo di un’estensione web: il file \textit{manifest.json}, il \textit{service worker} \textit{(background script)} e il \textit{content script}. In particolare, ho trovato stimolante poter applicare e integrare le conoscenze acquisite durante il percorso accademico - come API, design pattern, testing e Chrome DevTools - all’interno di un ambiente di sviluppo moderno, quello delle estensioni.

\vspace{10pt}
\par\noindent Il progetto mi ha permesso di analizzare in profondità il linguaggio \gls{javascript}, che in passato avevo sempre messo in secondo piano rispetto ad \gls{html} e \gls{css}, imparando ad apprezzarne sia le potenzialità che le criticità. Inoltre, ho potuto mettere in pratica le competenze maturate durante il corso di Ingegneria del Software, applicando fin dall’inizio le pratiche di sviluppo in modo attento e consapevole. Questo approccio mi ha aiutato a organizzare il lavoro in modo sostenibile, rispettare le scadenze e comprendere concretamente i vantaggi di una buona progettazione nel lungo periodo, soprattutto nella fase di testing.

\vspace{10pt}
\par\noindent Tra le competenze più significative acquisite durante lo stage, vi è senza dubbio la capacità di integrare nuove funzionalità all’interno di un progetto preesistente, unita alla flessibilità necessaria per adattarsi all’architettura e allo stile definiti da altri sviluppatori. Contribuire a un software non scritto in prima persona ha rappresentato per me una sfida nuova e stimolante, considerando che, nel mio percorso accademico, avevo sempre lavorato a progetti sviluppati “da zero”.

\section{Valutazione personale}

\par Ho trovato questa esperienza estremamente formativa, perché mi ha dato la possibilità di sviluppare un progetto destinato a scenari d’uso concreti, e non esclusivamente orientato alla ricerca teorica, con la consapevolezza del contesto di utilizzo primario. L’estensione, infatti, è pensata come strumento di supporto per la Proponente nella valutazione dei progetti didattici realizzati dagli studenti iscritti al corso di Tecnologie Web. La definizione concreta del contesto di utilizzo mi ha motivato a seguire un modello di sviluppo il più possibile allineato allo “stato dell’arte”, cosa che in passato era avvenuta in modo meno rigoroso o solo parzialmente consapevole. Ritengo che il software realizzato possa costituire una risorsa utile anche per i miei futuri progetti personali.

\vspace{10pt}
\par\noindent Dal punto di vista prettamente teorico, il tirocinio mi ha avvicinato ulteriormente al mondo dell’ottimizzazione \gls{seo}, ambito che avevo già iniziato a esplorare collaborando con una redazione online nella scrittura di blog tramite \gls{wordpress}. Così come rendere i contenuti web accessibili è fondamentale, anche ottimizzarli per migliorarne il posizionamento sui motori di ricerca si è rivelata, grazie a questo progetto, un’attività non solo necessaria ma anche affascinante e intrigante, indipendentemente dal settore digitale di applicazione.
