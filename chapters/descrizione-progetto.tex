\chapter{Descrizione del progetto}
\label{cap:descrizione-progetto}

\section{Analisi preventiva dei rischi}
\label{sec:rischi}

\par Durante le discussioni iniziali con la Proponente, sono emersi alcuni rischi che, se non prevenuti o mitigati correttamente, possono compromettere l’esito del progetto e il rispetto delle scadenze. Di seguito sono illustrati i rischi identificati e le relative strategie di prevenzione e/o mitigazione.

\begin{risk}{Risultato non conforme alle aspettative}
  \riskdescription{trattandosi di uno stage interno, la Proponente ha evidenziato il rischio che, in caso di mancanza di feedback costanti, la soluzione sviluppata, in termini tecnici o di interfaccia grafica, non soddisfi le attese}
  \riskprevention{creazione di una cartella condivisa su Google Drive, all’interno della quale verrà caricata la documentazione del progetto, che la Proponente potrà monitorare e commentare.\\ Condivisione del repository GitHub per consentire l’esecuzione di test manuali in parallelo a quelli automatici.\\ Aggiornamento in presenza o da remoto almeno una o due volte a settimana. In caso di aggiornamento a distanza, la comunicazione via mail dovrà essere sufficientemente esaustiva}
  \label{risk:risultato-non-conforme} 
\end{risk}

\begin{risk}{Performance dell’estensione}
  \riskdescription{per migliorare l'affidabilità dell’analisi delle parole chiave, potrebbe essere necessario sviluppare algoritmi più complessi dal punto di vista computazionale. Una volta integrate le nuove funzionalità nel progetto esistente, le performance potrebbero non risultare ottimali, poiché quest’ultimo esegue già operazioni onerose, come l’analisi delle immagini}
  \riskprevention{prima di sviluppare algoritmi orientati a migliorare l’affidabilità a discapito delle prestazioni, lo stagista dovrà inviare alla Proponente una comunicazione via mail contenente un confronto, in termini di pro e contro, tra la soluzione attuale e quella proposta. Tale comunicazione dovrà includere esempi pratici basati su analisi e benchmarking, lasciando alla Proponente la decisione finale in funzione degli obiettivi del progetto}
  \riskmitigation{nel caso in cui il calo delle performance si verifichi comunque, lo stagista e la Proponente organizzeranno un incontro per individuare una soluzione che garantisca il giusto compromesso tra affidabilità e prestazioni}
  \label{risk:prestazioni} 
\end{risk}

\begin{risk}{Sottostima del tempo necessario}
  \riskdescription{il preventivo “a finire” potrebbe comportare uno sforamento della scadenza prevista}
  \riskmitigation{revisione e riformulazione dei requisiti obbligatori minimi, al fine di garantire il raggiungimento degli obiettivi previsti ed evitare sforamenti eccessivi}
  \label{risk:scadenze} 
\end{risk}

\begin{risk}{Integrazione con un progetto esistente}
  \riskdescription{l’integrazione della funzionalità di analisi delle parole chiave con un progetto esistente potrebbe risultare problematica, a causa di possibili incompatibilità legate all’architettura, alle performance o all’interfaccia grafica}
  \riskmitigation{per quanto l’integrazione con un progetto esistente sia preferibile, in modo da centralizzare le funzionalità in un unico strumento (che sarà utilizzato dalla Proponente come sistema di valutazione), lo stagista è libero di sviluppare una soluzione ad hoc qualora dovessero sorgere problemi nel processo di integrazione}
  \label{risk:integrazione} 
\end{risk}

\begin{risk}{Scarsa definizione dei requisiti iniziali}
  \riskdescription{se non formulati con attenzione a scenari d’uso concreti, i requisiti potrebbero richiedere diverse rielaborazioni nel corso del progetto, il che potrebbe comportare frequenti refactoring sostanziali}
  \riskprevention{la prima settimana e mezza di stage sarà dedicata all’analisi delle soluzioni esistenti e alla stesura formale degli obiettivi, al fine di garantire una buona stabilità dei requisiti}
  \label{risk:requisiti-imprecisi} 
\end{risk}

\begin{risk}{Apprendimento delle tecnologie}
  \riskdescription{Alcune tecnologie, legate principalmente allo sviluppo di estensioni web, presentano una curva di apprendimento che potrebbe rallentare l’avanzamento del progetto}
  \riskprevention{La prima settimana di stage sarà dedicata all’identificazione delle tecnologie necessarie allo sviluppo, mentre la seconda sarà riservata allo studio delle stesse}
  \label{risk:apprendimento-tecnologie} 
\end{risk}

\begin{risk}{Test automatici non esaustivi}
  \riskdescription{sebbene la funzionalità di analisi delle parole chiave possa disporre di una copertura completa dei test automatici, l’integrazione con un progetto esistente rende più difficile testare in modo automatizzato interazioni reali e complesse dal punto di vista dell’utente}
  \riskmitigation{al termine dello sviluppo, verrà inviato un questionario SUS a un campione di utenti, al fine di raccogliere ulteriori feedback che, pur essendo orientati all’usabilità, possono offrire un contributo aggiuntivo alla valutazione manuale del sistema}
  \label{risk:test-automatici} 
\end{risk}

\section{Pianificazione del lavoro}
\label{sec:pianificazione}

\par Il periodo di stage è stato suddiviso in tre fasi principali:
\begin{itemize}
  \item \textbf{Analisi degli strumenti SEO} (dal 07/04/2025 al 21/04/2025);
  \item \textbf{Sviluppo delle funzionalità di analisi SEO} (dal 21/04/2025 al 02/06/2025);
  \item \textbf{Verifica e validazione del codice} (dal 19/05/2025 al 09/06/2025).
\end{itemize}

\subsection{Analisi degli strumenti SEO}

\par Il primo passo è stato identificare e analizzare gli strumenti di analisi SEO più diffusi sul mercato, selezionando quelli maggiormente in linea con le finalità del progetto. Da questa prima raccolta è stata effettuata una scrematura, con l’intento di concentrarsi soltanto sugli strumenti che offrissero funzionalità specifiche per l’analisi delle parole chiave o che presentassero un’interfaccia grafica accattivante. Questa fase ha ricoperto un ruolo fondamentale nella successiva definizione dei requisiti, permettendo di individuare sia le aree già coperte dal mercato, sia gli ambiti in cui introdurre soluzioni innovative. Inoltre, l’analisi dell’interfaccia grafica di ciascuno di questi strumenti ha fornito una base solida per la progettazione visiva dell’estensione. Dal punto di vista delle funzionalità, sono stati valutati i seguenti aspetti:

\begin{itemize}
  \item \textbf{Funzionalità complete e ampiamente diffuse}, come l’analisi del contrasto dei colori e il ranking delle keyword, già integrate nella maggior parte delle estensioni orientate all’accessibilità e all’ottimizzazione SEO. Poiché strumenti consolidati come Semrush, Ahrefs o Silktide sono attivi e supportati da anni, è difficile che le soluzioni sviluppate durante lo stage possano competere direttamente con essi. Pertanto, tali funzionalità sono state considerate già coperte;
  \item \textbf{Funzionalità incomplete o non particolarmente user-friendly}. Un esempio è l’analisi della distribuzione delle keyword, che diverse estensioni web implementano in modo superficiale, riportando soltanto la frequenza globale e trascurando il numero di occorrenze nei singoli tag. Inoltre, questa tipologia di analisi richiede spesso di interrompere la navigazione per aprire popup o accedere a piattaforme proprietarie;
  \item \textbf{Funzionalità poco diffuse e raramente supportate}. Un esempio è l’evidenziazione visiva della distribuzione delle keyword nella pagina, che la maggior parte delle estensioni non implementa. Quando presente, come nel caso di MozBar, questa funzionalità risulta comunque poco curata rispetto ad altre più consolidate.
\end{itemize}

\par\noindent Dopo aver ultimato l’analisi delle soluzioni esistenti e redatto la relativa documentazione, il tempo restante è stato dedicato all’analisi dei requisiti e alla realizzazione del mockup dell’interfaccia grafica. Sempre in questa fase, ho definito un workflow su GitHub, selezionato uno strumento di gestione del progetto e integrato i due ambienti. Inoltre, ho avviato lo studio delle tecnologie di sviluppo, nonché delle normative vigenti in materia di accessibilità e SEO.

\subsection{Sviluppo delle funzionalità di analisi SEO}

\par In questa fase ho sviluppato le funzionalità definite durante l’analisi dei requisiti e ho convertito in codice il mockup dell’interfaccia grafica. Lo sviluppo è stato suddiviso in due periodi: il primo dedicato alla realizzazione di un \gls{poc}, il secondo riservato alla progettazione, alla scelta dei design pattern, alla codifica e ai test. La versione dimostrativa è stata sviluppata in locale come pagina web standard, simulando un’estensione tramite una barra laterale e sostituendo una pagina reale con una pagina di analisi statica. Questo approccio ha permesso di disporre fin da subito di una demo funzionante da presentare alla Proponente, facilitando la raccolta di feedback sulle funzionalità e sull’interfaccia grafica. Una volta realizzato e convalidato il PoC, l’integrazione con il progetto esistente è risultata più naturale. Per garantire il rispetto delle scadenze, sono state fissate delle milestone, una per ogni versione rilasciata a partire dal PoC.

\vspace{10pt}
\par\noindent Ho adottato un sistema di versionamento conforme al formato \textbf{Semantic Versioning}:

\[
X.Y.Z
\]

\begin{itemize}
  \item \textbf{X (MAJOR)}: introduce funzionalità o cambiamenti che rompono  la retrocompatibilità;
  \item \textbf{Y (MINOR)}: aggiunge funzionalità in modo retrocompatibile;
  \item \textbf{Z (PATCH)}: applica correzioni di bug o comportamenti imprevisti, sempre in modo retrocompatibile.
\end{itemize}

\subsection{Verifica e validazione del codice}

\par In parallelo con l’attività di codifica, ho avviato la fase di testing, comprendente sia test automatici che manuali. I test di unità e integrazione sono stati automatizzati grazie a GitHub Actions, così da ottenere feedback immediati al verificarsi di determinati eventi, come l’apertura, l’aggiornamento o la chiusura di una pull request. Questo approccio riduce il rischio di introdurre errori nel branch principale e garantisce una copertura uniforme del codice, grazie alle funzionalità di coverage messe a disposizione dai framework di test. I test manuali sono stati effettuati sia dalle parti direttamente coinvolte nello sviluppo, sia da soggetti esterni privi di familiarità con il progetto.

\vspace{20pt}
\par\noindent Come strumento di gestione del progetto ho scelto Jira, più articolato rispetto a Trello ma perfettamente integrabile con GitHub. Si tratta, inoltre, di uno strumento già utilizzato in ambito accademico, di cui conoscevo punti di forza e limitazioni. Le funzionalità maggiormente impiegate sono state:
\begin{itemize}
  \item \textbf{Epic}: per definire le tre macro-fasi del progetto;
  \item \textbf{Ticket}: per tracciare le attività e sotto-attività;
  \item \textbf{Automazioni}: per integrare il flusso di lavoro automatizzato di GitHub con quello di Jira.
\end{itemize}

\par\noindent Di seguito è riportata un’immagine della timeline Jira che ha guidato l’avanzamento del progetto.

\begin{figure}[H]
  \centering 
  \fbox{\includegraphics[width=0.8\columnwidth]{pianificazione/jira.png}} 
  \caption{Timeline Jira dal 07/04/2025 al 09/06/2025}
\end{figure}

\section{Strumenti e tecnologie}
\label{sec:strumenti-tecnologie}

\par Di seguito sono elencati, in ordine alfabetico, i principali strumenti e tecnologie utilizzati durante lo svolgimento del progetto.

\subsection{Strumenti}

\subsection*{Chrome DevTools}

\par Gli strumenti per sviluppatori di Chrome consentono di ispezionare, debuggare e modificare le pagine web. Sono integrati direttamente nel browser e fungono da console di debug, offrendo agli sviluppatori la possibilità di individuare rapidamente eventuali errori.

\begin{figure}[H]
    \centering 
    \includegraphics[width=0.1\columnwidth]{strumenti-tecnologie/devtools_logo.png} 
\end{figure}

\subsection*{Codecov}

\par Codecov è un servizio di reporting sulla copertura del codice, che monitora la percentuale di codice eseguito durante i test automatici. Nell’ambito del progetto di stage, è stato utilizzato per integrare queste informazioni direttamente nel flusso di lavoro su GitHub, in modo da garantire una copertura del codice uniforme per ogni pull request. Inoltre, Codecov fornisce un \textit{badge} di stato utile ad arricchire la documentazione del repository.

\begin{figure}[H]
    \centering 
    \includegraphics[width=0.1\columnwidth]{strumenti-tecnologie/codecov_logo.pdf} 
\end{figure}

\subsection*{Draw.io}

\par Draw.io è uno strumento web-based utilizzato per la creazione di sketch e diagrammi. La piattaforma è open source e mette a disposizione template personalizzabili. I progetti possono essere salvati localmente oppure online, grazie all’integrazione con Google Workspace (Google Drive) e Dropbox. In assenza di una connessione a Internet, è disponibile un'applicazione desktop per macOS, Windows e Linux. Nell’ambito del progetto di stage, Draw.io è stato impiegato per la realizzazione dei diagrammi dei casi d’uso e delle classi.

\begin{figure}[H]
    \centering 
    \includegraphics[width=0.2\columnwidth]{strumenti-tecnologie/draw_io_logo.pdf} 
\end{figure}

\subsection*{GitHub}

\par Github è un servizio web e cloud-based per l'archiviazione, il versionamento e la condivisione del codice. Nell’ambito dello stage, è stato utilizzato per contribuire allo sviluppo di un progetto esistente tramite la funzionalità di \textit{fork}, che consente di lavorare sul codice in modo isolato, per poi proporre le modifiche al repository originale. GitHub consente anche di aprire spazi di discussione relativi al codice tramite \textit{issue} e \textit{pull request}, oltre a fornire strumenti per la gestione del progetto come \textit{board} e \textit{milestone}. Inoltre, permette di automatizzare i flussi di lavoro attraverso le \textit{GitHub Actions}.

\begin{figure}[H]
    \centering 
    \includegraphics[width=0.1\columnwidth]{strumenti-tecnologie/github_logo.pdf} 
\end{figure}

\subsection*{Google Docs}

\par Google Docs è un software web-based per la scrittura e la condivisione di documenti. Nell’ambito del progetto di stage, è stato utilizzato per la redazione dei seguenti documenti: piano di lavoro, analisi degli strumenti SEO, analisi dei requisiti, soluzioni progettuali e implementative.

\begin{figure}[H]
    \centering 
    \includegraphics[width=0.1\columnwidth]{strumenti-tecnologie/google_docs_logo.pdf} 
\end{figure}

\subsection*{Google Drive}

\par Google Drive è uno strumento web-based, parte di Google Workspace, che consente di archiviare, organizzare, condividere e accedere in modo sicuro a file e cartelle ovunque, da qualsiasi dispositivo connesso a Internet. Nell’ambito del progetto di stage, Google Drive è stato adottato come “unica fonte di verità”, ovvero come raccolta centralizzata di tutto il materiale condiviso tra stagista e Proponente (piano di lavoro, appunti, link utili, analisi dei requisiti, soluzioni progettuali, diagrammi, ecc.).

\begin{figure}[H]
    \centering 
    \includegraphics[width=0.1\columnwidth]{strumenti-tecnologie/google_drive_logo.pdf} 
\end{figure}

\subsection*{Google Forms}

\par Google Forms è uno strumento web-based per la creazione di moduli, questionari, quiz e sondaggi. Nell’ambito del progetto di stage, è stato utilizzato per la creazione di un questionario \gls{sus} finalizzato alla valutazione dell’usabilità del software sviluppato.

\begin{figure}[H]
    \centering 
    \includegraphics[width=0.1\columnwidth]{strumenti-tecnologie/google_forms_logo.pdf} 
\end{figure}

\subsection*{Google Sheets}

\par Google Sheets è un software web-based per la creazione e la condivisione di fogli di calcolo. Nell’ambito del progetto di stage, è stato utilizzato per memorizzare le risposte al questionario \gls{sus} e per creare una tabella di calcolo dei punteggi.

\begin{figure}[H]
    \centering 
    \includegraphics[width=0.1\columnwidth]{strumenti-tecnologie/google_sheets_logo.pdf} 
\end{figure}

\subsection*{Jira}

\par Jira è un'applicazione software sviluppata da Atlassian che consente agli utenti di gestire progetti, monitorare le attività e automatizzare i flussi di lavoro. È uno degli strumenti di gestione agile più utilizzati, in particolare in contesti fortemente collaborativi, poiché risponde alle esigenze dei team di pianificare, monitorare, rilasciare e supportare software in modo sicuro. Nell’ambito dello stage, Jira è stato impiegato per la pianificazione e il monitoraggio delle tre fasi principali del progetto: analisi delle soluzioni esistenti, sviluppo e collaudo. L’integrazione con GitHub consente di tracciare i ticket direttamente nell’ambiente di sviluppo e di automatizzarne la gestione. Inoltre, Jira mette a disposizione una timeline che offre una panoramica immediata delle dipendenze tra le attività e del rispetto delle scadenze.

\begin{figure}[H]
    \centering 
    \includegraphics[width=0.2\columnwidth]{strumenti-tecnologie/jira_logo.pdf} 
\end{figure}

\subsection*{npm}

\par npm (Node Package Manager) è un gestore di pacchetti per JavaScript che consente di gestire le dipendenze di un progetto. Tutti i pacchetti sono definiti nel file di configurazione \textit{package.json}. Nell’ambito dello stage, npm è stato utilizzato per configurare i test automatizzati tramite Jest e per gestire il versionamento del progetto.

\begin{figure}[H]
    \centering 
    \includegraphics[width=0.15\columnwidth]{strumenti-tecnologie/npm_logo.pdf} 
\end{figure}

\subsection*{Visual Studio Code}

\par Visual Studio Code è un editor di codice sorgente, più leggero e flessibile rispetto a un ambiente di sviluppo integrato tradizionale. Combina la semplicità di un editor con strumenti avanzati per sviluppatori, come il completamento del codice e altre funzionalità di assistenza alla scrittura.

\begin{figure}[H]
    \centering 
    \includegraphics[width=0.1\columnwidth]{strumenti-tecnologie/vscode_logo.png} 
\end{figure}

\subsection{Tecnologie}

\subsection*{Chrome Extension APIs}

\par Le Chrome Extension APIs forniscono agli sviluppatori un insieme di interfacce che permettono di estendere e personalizzare le funzionalità del browser. Queste API sono accessibili da qualsiasi componente dell’estensione. Consentono di accedere ai dati del browser, interagire con le schede, modificare il contenuto delle pagine web, memorizzare informazioni, gestire eventi e scambiare messaggi tra i componenti dell’estensione, ad esempio tra gli script di background e i content script.

\begin{figure}[H]
    \centering 
    \includegraphics[width=0.15\columnwidth]{strumenti-tecnologie/chrome_extension_logo.png} 
\end{figure}

\subsection*{CSS}

\par CSS (Cascading Style Sheets) è un linguaggio utilizzato per definire lo stile e la formattazione di documenti scritti in linguaggi di markup come HTML o XML. Nell’ambito del progetto di stage, è stata adottata la versione CSS3.

\begin{figure}[H]
    \centering 
    \includegraphics[width=0.25\columnwidth]{strumenti-tecnologie/css_logo.png} 
\end{figure}

\subsection*{HTML}

\par HTML (Hypertext Markup Language) è un linguaggio di markup utilizzato per definire la struttura e i contenuti delle pagine web. Nell’ambito del progetto di stage, è stata adottata la versione HTML5.

\begin{figure}[H]
    \centering 
    \includegraphics[width=0.15\columnwidth]{strumenti-tecnologie/html_logo.png} 
\end{figure}

\subsection*{JavaScript}

\par JavaScript è un linguaggio di programmazione utilizzato per definire il comportamento e la logica delle pagine web. È un linguaggio multi-paradigma che può essere impiegato sia nella programmazione lato client che lato server (Node.js). JavaScript consente di aggiornare dinamicamente i contenuti HTML e gli stili CSS. Si integra perfettamente con framework di test come Jest e strumenti di documentazione come JSDoc. È inoltre il linguaggio di riferimento per lo sviluppo di estensioni web.

\begin{figure}[H]
    \centering 
    \includegraphics[width=0.15\columnwidth]{strumenti-tecnologie/js_logo.png} 
\end{figure}

\subsection*{Jest}

\par Jest è un framework di test per JavaScript, utilizzato per la progettazione, lo sviluppo e l’esecuzione di test di unità e di integrazione. Permette di definire suite di test efficienti e isolate, senza richiedere configurazioni complesse, anche quando viene integrato in progetti già avviati. Aggiungendo il flag \verb|--coverage|, è possibile generare un report sulla copertura del codice e inviare i risultati a strumenti di terze parti come Codecov. Nell’ambito del progetto di stage, Jest è stato utilizzato insieme alla libreria Jest DOM per ottenere una copertura completa del codice.

\begin{figure}[H]
    \centering 
    \includegraphics[width=0.1\columnwidth]{strumenti-tecnologie/jest_logo.pdf} 
\end{figure}

\subsection*{Librerie di icone}

\par Per l’interfaccia grafica sono state utilizzate icone provenienti da tre librerie:
\begin{itemize}
  \item \textbf{Font Awesome}: una vasta libreria di icone vettoriali gratuite e premium;
  \item \textbf{Heroicons}: una raccolta di icone SVG realizzate a mano dai creatori di Tailwind CSS;
  \item \textbf{Remix Icon}: una libreria open source di icone vettoriali progettate per designer e sviluppatori.
\end{itemize}

\vspace{5pt}
\begin{center}
  \begin{minipage}{0.3\columnwidth}
    \centering
    \includegraphics[width=0.6\columnwidth]{strumenti-tecnologie/heroicons_logo.pdf} 
  \end{minipage}
  \hfill
  \begin{minipage}{0.3\columnwidth}
    \centering
    \includegraphics[width=0.2\columnwidth]{strumenti-tecnologie/font_awesome_logo.pdf} 
  \end{minipage}
  \hfill
  \begin{minipage}{0.3\columnwidth}
    \centering
    \includegraphics[width=0.6\columnwidth]{strumenti-tecnologie/remixicon_logo.pdf} 
  \end{minipage}
\end{center}

\subsection*{Manifest (manifest.json)}

\par Il file \textit{manifest.json} svolge un ruolo cruciale nello sviluppo di estensioni web, poiché consente agli sviluppatori di definire le funzionalità dell’estensione e le autorizzazioni richieste. Si tratta di un file di configurazione in formato \gls{json} che specifica una serie di metadati, tra cui i permessi, le risorse necessarie, gli script di background e i content script.

\subsection*{Stopword (stopword.js)}

\par Stopword è un modulo JavaScript che consente di rimuovere le \gls{stopword} da un testo. Oltre alla funzione \textit{removeStopwords}, l’API mette a disposizione identificatori univoci per accedere agli array di stopword associati a lingue specifiche, con supporto verificato per oltre 60 lingue. Il modulo è distribuito con licenza MIT.