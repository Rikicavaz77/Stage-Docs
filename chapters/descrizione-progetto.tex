\chapter{Descrizione del progetto}
\label{cap:descrizione-progetto}

\section{Pianificazione del lavoro}

\section{Strumenti e tecnologie}

\par Di seguito sono elencati, in ordine alfabetico, i principali strumenti e tecnologie utilizzati durante lo svolgimento del progetto.

\subsection{Strumenti}

\subsection*{Chrome DevTools}

\par Gli strumenti per sviluppatori di Chrome consentono di ispezionare, debuggare e modificare le pagine web. Sono integrati direttamente nel browser e fungono da console di debug, offrendo agli sviluppatori la possibilità di individuare rapidamente eventuali errori.

\begin{figure}[H]
    \centering 
    \includegraphics[width=0.1\columnwidth]{strumenti-tecnologie/devtools_logo.png} 
\end{figure}

\subsection*{Codecov}

\par Codecov è un servizio di reporting sulla copertura del codice, che monitora la percentuale di codice eseguito durante i test automatici. Nell’ambito del progetto di stage, è stato utilizzato per integrare queste informazioni direttamente nel flusso di lavoro su GitHub, in modo da garantire una copertura del codice uniforme per ogni pull request. Inoltre, Codecov fornisce un \textit{badge} di stato utile ad arricchire la documentazione del repository.

\begin{figure}[H]
    \centering 
    \includegraphics[width=0.1\columnwidth]{strumenti-tecnologie/codecov_logo.pdf} 
\end{figure}

\subsection*{Draw.io}

\par Draw.io è uno strumento web-based utilizzato per la creazione di sketch e diagrammi. La piattaforma è open source e mette a disposizione template personalizzabili. I progetti possono essere salvati localmente oppure online, grazie all’integrazione con Google Workspace (Google Drive) e Dropbox. In assenza di una connessione a Internet, è disponibile un'applicazione desktop per macOS, Windows e Linux. Nell’ambito del progetto di stage, Draw.io è stato impiegato per la realizzazione dei diagrammi dei casi d’uso e delle classi.

\begin{figure}[H]
    \centering 
    \includegraphics[width=0.2\columnwidth]{strumenti-tecnologie/draw_io_logo.pdf} 
\end{figure}

\subsection*{GitHub}

\par Github è un servizio web e cloud-based per l'archiviazione, il versionamento e la condivisione del codice. Nell’ambito dello stage, è stato utilizzato per contribuire allo sviluppo di un progetto esistente tramite la funzionalità di \textit{fork}, che consente di lavorare sul codice in modo isolato, per poi proporre le modifiche al repository originale. GitHub consente anche di aprire spazi di discussione relativi al codice tramite \textit{issue} e \textit{pull request}, oltre a fornire strumenti per la gestione del progetto come \textit{board} e \textit{milestone}. Inoltre, permette di automatizzare i flussi di lavoro attraverso le \textit{GitHub Actions}.

\begin{figure}[H]
    \centering 
    \includegraphics[width=0.1\columnwidth]{strumenti-tecnologie/github_logo.pdf} 
\end{figure}

\subsection*{Google Docs}

\par Google Docs è un software web-based per la scrittura e la condivisione di documenti. Nell’ambito del progetto di stage, è stato utilizzato per la redazione dei seguenti documenti: piano di lavoro, analisi degli strumenti SEO, analisi dei requisiti, soluzioni progettuali e implementative.

\begin{figure}[H]
    \centering 
    \includegraphics[width=0.1\columnwidth]{strumenti-tecnologie/google_docs_logo.pdf} 
\end{figure}

\subsection*{Google Drive}

\par Google Drive è uno strumento web-based, parte di Google Workspace, che consente di archiviare, organizzare, condividere e accedere in modo sicuro a file e cartelle ovunque, da qualsiasi dispositivo connesso a Internet. Nell’ambito del progetto di stage, Google Drive è stato adottato come “unica fonte di verità”, ovvero come raccolta centralizzata di tutto il materiale condiviso tra stagista e Proponente (piano di lavoro, appunti, link utili, analisi dei requisiti, soluzioni progettuali, diagrammi, ecc.).

\begin{figure}[H]
    \centering 
    \includegraphics[width=0.1\columnwidth]{strumenti-tecnologie/google_drive_logo.pdf} 
\end{figure}

\subsection*{Google Forms}

\par Google Forms è uno strumento web-based per la creazione di moduli, questionari, quiz e sondaggi. Nell’ambito del progetto di stage, è stato utilizzato per la creazione di un questionario \gls{sus} finalizzato alla valutazione dell’usabilità del software sviluppato.

\begin{figure}[H]
    \centering 
    \includegraphics[width=0.1\columnwidth]{strumenti-tecnologie/google_forms_logo.pdf} 
\end{figure}

\subsection*{Google Sheets}

\par Google Sheets è un software web-based per la creazione e la condivisione di fogli di calcolo. Nell’ambito del progetto di stage, è stato utilizzato per memorizzare le risposte al questionario \gls{sus} e per creare una tabella di calcolo dei punteggi.

\begin{figure}[H]
    \centering 
    \includegraphics[width=0.1\columnwidth]{strumenti-tecnologie/google_sheets_logo.pdf} 
\end{figure}

\subsection*{Jira}

\par Jira è un'applicazione software sviluppata da Atlassian che consente agli utenti di gestire progetti, monitorare le attività e automatizzare i flussi di lavoro. È uno degli strumenti di gestione agile più utilizzati, in particolare in contesti fortemente collaborativi, poiché risponde alle esigenze dei team di pianificare, monitorare, rilasciare e supportare software in modo sicuro. Nell’ambito dello stage, Jira è stato impiegato per la pianificazione e il monitoraggio delle tre fasi principali del progetto: analisi delle soluzioni esistenti, sviluppo e collaudo. L’integrazione con GitHub consente di tracciare i ticket direttamente nell’ambiente di sviluppo e di automatizzarne la gestione. Inoltre, Jira mette a disposizione una timeline che offre una panoramica immediata delle dipendenze tra le attività e del rispetto delle scadenze.

\begin{figure}[H]
    \centering 
    \includegraphics[width=0.2\columnwidth]{strumenti-tecnologie/jira_logo.pdf} 
\end{figure}

\subsection*{npm}

\par npm (Node Package Manager) è un gestore di pacchetti per JavaScript che consente di gestire le dipendenze di un progetto. Tutti i pacchetti sono definiti nel file di configurazione \textit{package.json}. Nell’ambito dello stage, npm è stato utilizzato per configurare i test automatizzati tramite Jest e per gestire il versionamento del progetto.

\begin{figure}[H]
    \centering 
    \includegraphics[width=0.15\columnwidth]{strumenti-tecnologie/npm_logo.pdf} 
\end{figure}

\subsection*{Visual Studio Code}

\par Visual Studio Code è un editor di codice sorgente, più leggero e flessibile rispetto a un ambiente di sviluppo integrato tradizionale. Combina la semplicità di un editor con strumenti avanzati per sviluppatori, come il completamento del codice e altre funzionalità di assistenza alla scrittura.

\begin{figure}[H]
    \centering 
    \includegraphics[width=0.1\columnwidth]{strumenti-tecnologie/vscode_logo.png} 
\end{figure}

\subsection{Tecnologie}

\subsection*{Chrome Extension APIs}

\par Le Chrome Extension APIs forniscono agli sviluppatori un insieme di interfacce che permettono di estendere e personalizzare le funzionalità del browser. Queste API sono accessibili da qualsiasi componente dell’estensione. Consentono di accedere ai dati del browser, interagire con le schede, modificare il contenuto delle pagine web, memorizzare informazioni, gestire eventi e scambiare messaggi tra i componenti dell’estensione, ad esempio tra gli script di background e i content script.

\begin{figure}[H]
    \centering 
    \includegraphics[width=0.15\columnwidth]{strumenti-tecnologie/chrome_extension_logo.png} 
\end{figure}

\subsection*{CSS}

\par CSS (Cascading Style Sheets) è un linguaggio utilizzato per definire lo stile e la formattazione di documenti scritti in linguaggi di markup come HTML o XML. Nell’ambito del progetto di stage, è stata adottata la versione CSS3.

\begin{figure}[H]
    \centering 
    \includegraphics[width=0.25\columnwidth]{strumenti-tecnologie/css_logo.png} 
\end{figure}

\subsection*{HTML}

\par HTML (Hypertext Markup Language) è un linguaggio di markup utilizzato per definire la struttura e i contenuti delle pagine web. Nell’ambito del progetto di stage, è stata adottata la versione HTML5.

\begin{figure}[H]
    \centering 
    \includegraphics[width=0.15\columnwidth]{strumenti-tecnologie/html_logo.png} 
\end{figure}

\subsection*{JavaScript}

\par JavaScript è un linguaggio di programmazione utilizzato per definire il comportamento e la logica delle pagine web. È un linguaggio multi-paradigma che può essere impiegato sia nella programmazione lato client che lato server (Node.js). JavaScript consente di aggiornare dinamicamente i contenuti HTML e gli stili CSS. Si integra perfettamente con framework di test come Jest e strumenti di documentazione come JSDoc. È inoltre il linguaggio di riferimento per lo sviluppo di estensioni web.

\begin{figure}[H]
    \centering 
    \includegraphics[width=0.15\columnwidth]{strumenti-tecnologie/js_logo.png} 
\end{figure}

\subsection*{Jest}

\par Jest è un framework di test per JavaScript, utilizzato per la progettazione, lo sviluppo e l’esecuzione di test di unità e di integrazione. Permette di definire suite di test efficienti e isolate, senza richiedere configurazioni complesse, anche quando viene integrato in progetti già avviati. Aggiungendo il flag \verb|--coverage|, è possibile generare un report sulla copertura del codice e inviare i risultati a strumenti di terze parti come Codecov. Nell’ambito del progetto di stage, Jest è stato utilizzato insieme alla libreria Jest DOM per ottenere una copertura completa del codice.

\begin{figure}[H]
    \centering 
    \includegraphics[width=0.1\columnwidth]{strumenti-tecnologie/jest_logo.pdf} 
\end{figure}

\subsection*{Librerie di icone}

\par Per l’interfaccia grafica sono state utilizzate icone provenienti da tre librerie:
\begin{itemize}
  \item \textbf{Font Awesome}: una vasta libreria di icone vettoriali gratuite e premium;
  \item \textbf{Heroicons}: una raccolta di icone SVG realizzate a mano dai creatori di Tailwind CSS;
  \item \textbf{Remix Icon}: una libreria open source di icone vettoriali progettate per designer e sviluppatori.
\end{itemize}

\vspace{5pt}
\begin{center}
  \begin{minipage}{0.3\columnwidth}
    \centering
    \includegraphics[width=0.6\columnwidth]{strumenti-tecnologie/heroicons_logo.pdf} 
  \end{minipage}
  \hfill
  \begin{minipage}{0.3\columnwidth}
    \centering
    \includegraphics[width=0.2\columnwidth]{strumenti-tecnologie/font_awesome_logo.pdf} 
  \end{minipage}
  \hfill
  \begin{minipage}{0.3\columnwidth}
    \centering
    \includegraphics[width=0.6\columnwidth]{strumenti-tecnologie/remixicon_logo.pdf} 
  \end{minipage}
\end{center}

\subsection*{Manifest (manifest.json)}

\par Il file \textit{manifest.json} svolge un ruolo cruciale nello sviluppo di estensioni web, poiché consente agli sviluppatori di definire le funzionalità dell’estensione e le autorizzazioni richieste. Si tratta di un file di configurazione in formato \gls{json} che specifica una serie di metadati, tra cui i permessi, le risorse necessarie, gli script di background e i content script.

\subsection*{Stopword (stopword.js)}

\par Stopword è un modulo JavaScript che consente di rimuovere le \gls{stopword} da un testo. Oltre alla funzione \textit{removeStopwords}, l’API mette a disposizione identificatori univoci per accedere agli array di stopword associati a lingue specifiche, con supporto verificato per oltre 60 lingue. Il modulo è distribuito con licenza MIT.