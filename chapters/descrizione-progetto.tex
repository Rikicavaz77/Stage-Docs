\chapter{Descrizione del progetto}
\label{cap:descrizione-progetto}

\section{Pianificazione del lavoro}

\section{Strumenti e tecnologie}

\par Di seguito sono elencati, in ordine alfabetico, i principali strumenti e tecnologie utilizzate durante lo svolgimento del progetto.

\subsection{Strumenti}

\subsection*{Draw.io}

\par Draw.io è uno strumento web-based utilizzato per la creazione di sketch e diagrammi. La piattaforma è open source e mette a disposizione template personalizzabili. I progetti possono essere salvati localmente oppure online, grazie all’integrazione con Google Workspace (Google Drive) e Dropbox. In assenza di una connessione a Internet, è disponibile un'applicazione desktop per macOS, Windows e Linux. Nell’ambito del progetto di stage, Draw.io è stato impiegato per la realizzazione dei diagrammi dei casi d’uso e delle classi.

\subsection*{Google Docs}

\par Google Docs è un software web-based per la scrittura e la condivisione di documenti. Nell’ambito del progetto di stage, è stato utilizzato per la redazione dei seguenti documenti: piano di lavoro, analisi degli strumenti SEO, analisi dei requisiti, soluzioni progettuali e implementative.

\subsection*{Google Drive}

\par Google Drive è uno strumento web-based, parte di Google Workspace, che consente di archiviare, organizzare, condividere e accedere in modo sicuro a file e cartelle ovunque, da qualsiasi dispositivo connesso a Internet. Nell’ambito del progetto di stage, Google Drive è stato adottato come “unica fonte di verità”, ovvero come raccolta centralizzata di tutto il materiale condiviso tra stagista e Proponente (piano di lavoro, appunti, link utili, analisi dei requisiti, soluzioni progettuali, diagrammi, ecc.).

\subsection*{Google Forms}

\par Google Forms è uno strumento web-based per la creazione di moduli, questionari, quiz e sondaggi. Nell’ambito del progetto di stage, è stato utilizzato per la creazione di un questionario \gls{sus} finalizzato alla valutazione dell’usabilità del software sviluppato.

\subsection*{Google Sheets}

\par Google Sheets è un software web-based per la creazione e la condivisione di fogli di calcolo. Nell’ambito del progetto di stage, è stato utilizzato per memorizzare le risposte al questionario \gls{sus} e per creare una tabella di calcolo dei punteggi.

\subsection*{Jira}

\par Jira è un'applicazione software sviluppata da Atlassian che consente agli utenti di gestire progetti, monitorare le attività e automatizzare i flussi di lavoro. È uno degli strumenti di gestione agile più utilizzati, in particolare in contesti fortemente collaborativi, poiché risponde alle esigenze dei team di pianificare, monitorare, rilasciare e supportare software in modo sicuro. Nell’ambito dello stage, Jira è stato impiegato per la pianificazione e il monitoraggio delle tre fasi principali del progetto: analisi delle soluzioni esistenti, sviluppo e collaudo. L’integrazione con GitHub consente di tracciare i ticket direttamente nell’ambiente di sviluppo e di automatizzarne la gestione. Inoltre, Jira mette a disposizione una timeline che offre una panoramica immediata delle dipendenze tra le attività e del rispetto delle scadenze.

