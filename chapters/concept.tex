\chapter{Analisi dei requisiti}
\label{cap:analisi-requisiti}

\par L'analisi dei \gls{requisiti} è stata condotta seguendo gli standard dell'ingegneria del software, integrando l'analisi del progetto e delle soluzioni esistenti con colloqui e dialoghi con la Proponente. In questa sezione sono illustrati i seguenti punti:
\begin{itemize}
    \item \textbf{\gls{use-case}}: descrizione e diagrammi dei casi d'uso;
    \item \textbf{Tracciamento dei requisiti};
    \item \textbf{\gls{user-story}}.
\end{itemize}

\section{Casi d'uso}

\paragraph*{Attori}
\par L'unico attore coinvolto nell'interazione con il sistema è un \textbf{utente generico} con accesso completo alla funzionalità di analisi \gls{seog}. Di seguito sono elencate alcune tipologie di utenti a cui è rivolto il progetto:
\begin{itemize}
    \item \textbf{Sviluppatore}: utilizza l'estensione durante lo sviluppo e la produzione di contenuti web;
    \item \textbf{Tester}: effettua un'analisi SEO per identificare eventuali problemi e proporre azioni di miglioramento;
    \item \textbf{Professore}: utilizza l'estensione per analizzare progetti didattici.
\end{itemize}

\begin{usecase}{1}{Accesso alla funzionalità di analisi delle parole chiave}\label{UC1}
    \usecaseactors{Utente.}
    \usecasepre{L'utente ha avviato l'estensione.}
    \usecasedesc{L'utente seleziona la funzionalità di analisi delle parole chiave.}
    \usecasepost{Il sistema mostra la schermata di analisi delle parole chiave.}
\end{usecase}

\begin{usecase}{2}{Visualizzazione di una panoramica dell'analisi delle parole chiave}\label{UC2}
    \usecaseactors{Utente.}
    \usecasepreEnv{
        \begin{itemize}
            \item L'utente ha selezionato la funzionalità di analisi delle parole chiave;
            \item Il sistema è attivo e funzionante.
        \end{itemize}}
    \usecasepost{Il sistema mostra una panoramica dell'analisi delle parole chiave.}
    \usecasesubEnv{
        \begin{itemize}
            \item \hyperref[UC2point1]{UC2.1}: Visualizzazione del meta tag `keywords`;
            \item \hyperref[UC2point2]{UC2.2}: Visualizzazione del numero totale di parole nella pagina;
            \item \hyperref[UC2point3]{UC2.3}: Visualizzazione del numero di parole uniche nella pagina;
            \item \hyperref[UC2point4]{UC2.4}: Visualizzazione della lingua della pagina.
        \end{itemize}}
\end{usecase}

\begin{usecase}{2.1}{Visualizzazione del meta tag `keywords`}\label{UC2point1}
    \usecaseactors{Utente.}
    \usecasepreEnv{\begin{itemize}
        \item L'utente ha selezionato la funzionalità di analisi delle parole chiave;
        \item Il sistema è attivo e funzionante.
    \end{itemize}}
    \usecasepost{Il sistema mostra il meta tag `keywords`.}
    \usecaseextEnv{
        \begin{itemize}
            \item Visualizzazione di un avviso se il meta tag `keywords` non è presente (\hyperref[UC10]{UC10}).
        \end{itemize}}
\end{usecase}

\begin{usecase}{2.2}{Visualizzazione del numero totale di parole nella pagina}\label{UC2point2}
    \usecaseactors{Utente.}
    \usecasepreEnv{\begin{itemize}
        \item L'utente ha selezionato la funzionalità di analisi delle parole chiave;
        \item Il sistema è attivo e funzionante.
    \end{itemize}}
    \usecasepost{Il sistema mostra il numero totale di parole presenti nella pagina.}
\end{usecase}

\begin{usecase}{2.3}{Visualizzazione del numero di parole uniche nella pagina}\label{UC2point3}
    \usecaseactors{Utente.}
    \usecasepreEnv{\begin{itemize}
        \item L'utente ha selezionato la funzionalità di analisi delle parole chiave;
        \item Il sistema è attivo e funzionante.
    \end{itemize}}
    \usecasepost{Il sistema mostra il numero totale di parole uniche presenti nella pagina.}
\end{usecase}

\begin{usecase}{2.4}{Visualizzazione della lingua della pagina}\label{UC2point4}
    \usecaseactors{Utente.}
    \usecasepreEnv{\begin{itemize}
        \item L'utente ha selezionato la funzionalità di analisi delle parole chiave;
        \item Il sistema è attivo e funzionante.
    \end{itemize}}
    \usecasepost{Il sistema mostra la lingua della pagina.}
\end{usecase}

\begin{usecase}{3}{Inserimento di una parola chiave}\label{UC3}
    \usecaseactors{Utente.}
    \usecasepreEnv{\begin{itemize}
        \item L'utente ha selezionato la funzionalità di analisi delle parole chiave;
        \item Il sistema è attivo e funzionante.
    \end{itemize}}
    \usecasedesc{L'utente digita una parola chiave da analizzare.}
    \usecasepost{Il sistema mostra la parola chiave inserita dall'utente nell'apposito campo di testo.}
\end{usecase}

\begin{usecase}{4}{Inserimento di una parola chiave}\label{UC4}
    \usecaseactors{Utente.}
    \usecasepreEnv{\begin{itemize}
        \item L'utente ha selezionato la funzionalità di analisi delle parole chiave;
        \item Il sistema è attivo e funzionante.
    \end{itemize}}
    \usecasedesc{Il sistema esegue l'analisi di una parola chiave scelta dall'utente.}
    \usecasepost{Il sistema aggiunge la parola chiave analizzata alla lista delle keyword.}
    \usecaseincEnv{\begin{itemize}
        \item Inserimento di una parola chiave (\hyperref[UC3]{UC3}).
    \end{itemize}}
\end{usecase}

\begin{usecase}{5}{Visualizzazione di una lista delle parole chiave}\label{UC5}
    \usecaseactors{Utente.}
    \usecasepreEnv{\begin{itemize}
        \item L'utente ha selezionato la funzionalità di analisi delle parole chiave;
        \item Il sistema è attivo e funzionante.
    \end{itemize}}
    \usecasepost{Il sistema mostra una lista delle parole chiave.}
    \usecasesubEnv{
        \begin{itemize}
            \item \hyperref[UC5point1]{UC5.1}: Visualizzazione di una lista delle parole chiave estratte dal meta tag `keywords`;
            \item \hyperref[UC5point2]{UC5.2}: Visualizzazione di una lista delle parole chiave inserite dall'utente;
            \item \hyperref[UC5point3]{UC5.3}: Visualizzazione di una lista delle parole chiave più frequenti.
        \end{itemize}}
    \usecaseincEnv{\begin{itemize}
        \item Visualizzazione di una singola parola chiave (\hyperref[UC6]{UC6}).
    \end{itemize}}
\end{usecase}

\begin{usecase}{5.1}{Visualizzazione di una lista delle parole chiave estratte dal meta tag `keywords`}\label{UC5point1}
    \usecaseactors{Utente.}
    \usecasepreEnv{\begin{itemize}
        \item L'utente ha selezionato la funzionalità di analisi delle parole chiave;
        \item Il sistema è attivo e funzionante.
    \end{itemize}}
    \usecasepost{Il sistema mostra una lista delle parole chiave estratte dal meta tag `keywords`.}
\end{usecase}

\begin{usecase}{5.2}{Visualizzazione di una lista delle parole chiave inserite dall'utente}\label{UC5point2}
    \usecaseactors{Utente.}
    \usecasepreEnv{\begin{itemize}
        \item L'utente ha selezionato la funzionalità di analisi delle parole chiave;
        \item Il sistema è attivo e funzionante.
    \end{itemize}}
    \usecasepost{Il sistema mostra una lista delle parole chiave inserite dall'utente.}
\end{usecase}

\begin{usecase}{5.3}{Visualizzazione di una lista delle parole chiave più frequenti}\label{UC5point3}
    \usecaseactors{Utente.}
    \usecasepreEnv{\begin{itemize}
        \item L'utente ha selezionato la funzionalità di analisi delle parole chiave;
        \item Il sistema è attivo e funzionante.
    \end{itemize}}
    \usecasepost{Il sistema mostra una lista delle parole chiave più frequenti.}
\end{usecase}

\begin{usecase}{6}{Visualizzazione di una singola parola chiave}\label{UC6}
    \usecaseactors{Utente.}
    \usecasepreEnv{\begin{itemize}
        \item L'utente ha selezionato la funzionalità di analisi delle parole chiave;
        \item Il sistema è attivo e funzionante;
        \item È visibile la lista delle parole chiave.
    \end{itemize}}
    \usecasepost{L'utente visualizza una singola parola chiave.}
    \usecaseincEnv{\begin{itemize}
        \item Visualizzazione dei risultati dell'analisi di una parola chiave (\hyperref[UC7]{UC7}).
    \end{itemize}}
\end{usecase}

\begin{usecase}{7}{Visualizzazione dei risultati dell'analisi di una parola chiave}\label{UC7}
    \usecaseactors{Utente.}
    \usecasepreEnv{\begin{itemize}
        \item L'utente ha selezionato la funzionalità di analisi delle parole chiave;
        \item Il sistema è attivo e funzionante.
    \end{itemize}}
    \usecasepost{L'utente visualizza i risultati dell'analisi di una parola chiave.}
    \usecasesubEnv{
        \begin{itemize}
            \item \hyperref[UC7point1]{UC7.1}: Visualizzazione della frequenza di una parola chiave;
            \item \hyperref[UC7point2]{UC7.2}: Visualizzazione della densità di una parola chiave;
            \item \hyperref[UC7point3]{UC7.3}: Visualizzazione di un elenco dei tag in cui dovrebbe essere presente una parola chiave.
        \end{itemize}}
\end{usecase}

\begin{usecase}{7.1}{Visualizzazione della frequenza di una parola chiave}\label{UC7point1}
    \usecaseactors{Utente.}
    \usecasepreEnv{\begin{itemize}
        \item L'utente ha selezionato la funzionalità di analisi delle parole chiave;
        \item Il sistema è attivo e funzionante.
    \end{itemize}}
    \usecasepost{L'utente visualizza la frequenza di una parola chiave.}
\end{usecase}

\begin{usecase}{7.2}{Visualizzazione della densità di una parola chiave}\label{UC7point2}
    \usecaseactors{Utente.}
    \usecasepreEnv{\begin{itemize}
        \item L'utente ha selezionato la funzionalità di analisi delle parole chiave;
        \item Il sistema è attivo e funzionante.
    \end{itemize}}
    \usecasepost{L'utente visualizza la densità di una parola chiave.}
    \usecaseextEnv{\begin{itemize}
        \item Visualizzazione di un avviso se la densità di una parola chiave è troppo alta (\hyperref[UC11]{UC11}).
    \end{itemize}}
\end{usecase}

\begin{usecase}{7.3}{Visualizzazione di un elenco dei tag in cui dovrebbe essere presente una parola chiave}\label{UC7point3}
    \usecaseactors{Utente.}
    \usecasepreEnv{\begin{itemize}
        \item L'utente ha selezionato la funzionalità di analisi delle parole chiave;
        \item Il sistema è attivo e funzionante.
    \end{itemize}}
    \usecasepost{L'utente visualizza un elenco dei tag in cui dovrebbe essere presente una parola chiave.}
    \usecasesubEnv{\begin{itemize}
        \item \hyperref[UC7point3point1]{UC7.3.1}: Visualizzazione di un singolo tag.
    \end{itemize}}
\end{usecase}

\begin{usecase}{7.3.1}{Visualizzazione di un singolo tag}\label{UC7point3point1}
    \usecaseactors{Utente.}
    \usecasepreEnv{\begin{itemize}
        \item L'utente ha selezionato la funzionalità di analisi delle parole chiave;
        \item Il sistema è attivo e funzionante;
        \item L'utente sta visualizzando i risultati dell'analisi di una parola chiave;
        \item È visibile l'elenco dei tag.
    \end{itemize}}
    \usecasepost{L'utente visualizza un singolo tag.}
    \usecasesubEnv{\begin{itemize}
        \item \hyperref[UC7point3point1point1]{UC7.3.1.1}: Visualizzazione del nome del tag;
        \item \hyperref[UC7point3point1point2]{UC7.3.1.2}: Visualizzazione del numero di occorrenze di una parola chiave nel tag.
    \end{itemize}}
\end{usecase}

\begin{usecase}{7.3.1.1}{Visualizzazione del nome del tag}\label{UC7point3point1point1}
    \usecaseactors{Utente.}
    \usecasepreEnv{\begin{itemize}
        \item L'utente ha selezionato la funzionalità di analisi delle parole chiave;
        \item Il sistema è attivo e funzionante.
    \end{itemize}}
    \usecasepost{L'utente visualizza il nome del tag.}
\end{usecase}

\begin{usecase}{7.3.1.2}{Visualizzazione del numero di occorrenze di una parola chiave nel tag}\label{UC7point3point1point2}
    \usecaseactors{Utente.}
    \usecasepreEnv{\begin{itemize}
        \item L'utente ha selezionato la funzionalità di analisi delle parole chiave;
        \item Il sistema è attivo e funzionante.
    \end{itemize}}
    \usecasepost{L'utente visualizza il numero di occorrenze di una parola chiave nel tag.}
    \usecaseextEnv{\begin{itemize}
        \item Visualizzazione di un avviso se il numero di occorrenze di una parola chiave nel tag è inferiore a una certa soglia (\hyperref[UC12]{UC12}).
    \end{itemize}}
\end{usecase}

\begin{usecase}{8}{Evidenziazione di una parola chiave nella pagina}\label{UC8}
    \usecaseactors{Utente.}
    \usecasepreEnv{\begin{itemize}
        \item L'utente ha selezionato la funzionalità di analisi delle parole chiave;
        \item Il sistema è attivo e funzionante.
    \end{itemize}}
    \usecasepost{Il sistema evidenzia graficamente tutte le occorrenze di una parola chiave nella pagina.}
\end{usecase}

\begin{usecase}{9}{Aggiornamento dell'analisi delle parole chiave}\label{UC9}
    \usecaseactors{Utente.}
    \usecasepreEnv{\begin{itemize}
        \item L'utente ha selezionato la funzionalità di analisi delle parole chiave;
        \item Il sistema è attivo e funzionante.
    \end{itemize}}
    \usecasedesc{L'utente aggiorna manualmente l'analisi delle parole chiave per sincronizzarla con eventuali modifiche dinamiche al DOM.}
    \usecasepost{Il sistema aggiorna l'analisi delle parole chiave.}
\end{usecase}

\begin{usecase}{10}{Visualizzazione di un avviso se il meta tag `keywords` non è presente}\label{UC10}
    \usecaseactors{Utente.}
    \usecasepreEnv{\begin{itemize}
        \item L'utente ha selezionato la funzionalità di analisi delle parole chiave;
        \item Il sistema è attivo e funzionante.
    \end{itemize}}
    \usecasepost{Il sistema mostra un avviso in cui notifica all'utente che il meta tag `keywords` non è presente.}
\end{usecase}

\begin{usecase}{11}{Visualizzazione di un avviso se la densità di una parola chiave è troppo alta}\label{UC11}
    \usecaseactors{Utente.}
    \usecasepreEnv{\begin{itemize}
        \item L'utente ha selezionato la funzionalità di analisi delle parole chiave;
        \item Il sistema è attivo e funzionante.
    \end{itemize}}
    \usecasepost{Il sistema mostra un avviso in cui notifica all'utente che la densità di una parola chiave è troppo alta.}
\end{usecase}

\begin{usecase}{12}{Visualizzazione di un avviso se il numero di occorrenze di una parola chiave nel tag è inferiore a una certa soglia}\label{UC12}
    \usecaseactors{Utente.}
    \usecasepreEnv{\begin{itemize}
        \item L'utente ha selezionato la funzionalità di analisi delle parole chiave;
        \item Il sistema è attivo e funzionante.
    \end{itemize}}
    \usecasepost{Il sistema mostra un avviso in cui notifica all'utente che il numero di occorrenze di una parola chiave nel tag è inferiore a una certa soglia.}
\end{usecase}

\newpage

\section{Tracciamento dei requisiti}
\par I requisiti del progetto, individuati e formalizzati durante il processo di analisi, sono illustrati nelle tabelle \ref{tab:requisiti-funzionali}, \ref{tab:requisiti-qualitativi} e \ref{tab:requisiti-vincolo} secondo la seguente notazione:
\par \textbf{\[R[Tipologia].[Importanza].[Codice]\]} 
\par dove:
\par\vspace{20pt}
\begin{tabular}{@{}ll@{}}
    R = & requisito \\
    \textbf{Tipologia}: & \\
    \quad F = & funzionale \\
    \quad Q = & di qualità \\
    \quad V = & di vincolo/dominio \\
    \textbf{Importanza}: & \\
    \quad O = & obbligatorio \\
    \quad D = & desiderabile \\  
    \quad OP = & opzionale \\
    Codice = & codice numerico univoco \\
\end{tabular}
    
\par\vspace{30pt}

\renewcommand{\arraystretch}{1.5}
\begin{longtable}{p{0.15\textwidth}p{0.55\textwidth}p{0.2\textwidth}}
\caption{Tabella dei requisti funzionali}
\label{tab:requisiti-funzionali} \\
\hline\hline
\textbf{Requisito} & \textbf{Descrizione} & \textbf{Fonti}\\
\endfirsthead

\caption[]{Tabella dei requisiti funzionali (continua)} \\
\hline\hline
\textbf{Requisito} & \textbf{Descrizione} & \textbf{Fonti} \\ 
\endhead

\multicolumn{3}{r}{{Continua nella prossima pagina}} \\ 
\endfoot

\hline
\endlastfoot

\hline
RF.O.1 & L'utente deve poter accedere alla funzionalità di analisi delle parole chiave. & \hyperref[UC1]{UC1} \\
\hline
RF.O.2 & L'utente deve poter visualizzare una panoramica dell'analisi delle parole chiave. & \hyperref[UC2]{UC2} \\
\hline
RF.O.3 & L'utente deve poter visualizzare il meta tag `keywords`. & \hyperref[UC2point1]{UC2.1} \\
\hline
RF.O.4 & L'utente deve poter visualizzare il numero totale di parole nella pagina. & \hyperref[UC2point2]{UC2.2} \\
\hline
RF.OP.5 & L'utente deve poter visualizzare il numero di parole uniche nella pagina. & \hyperref[UC2point3]{UC2.3} \\
\hline
RF.O.6 & L'utente deve poter visualizzare la lingua della pagina. & \hyperref[UC2point4]{UC2.4} \\
\hline
RF.O.7 & L'utente deve poter inserire una parola chiave da analizzare. & \hyperref[UC3]{UC3} \\
\hline
RF.O.8 & L'utente deve poter analizzare una parola chiave. & \hyperref[UC4]{UC4} \\
\hline
RF.O.9 & L'utente deve poter visualizzare una lista delle parole chiave. & \hyperref[UC5]{UC5} \\
\hline
RF.O.10 & L'utente deve poter visualizzare una lista delle parole chiave estratte dal meta tag `keywords`. & \hyperref[UC5point1]{UC5.1} \\
\hline
RF.O.11 & Il sistema deve visualizzare una lista delle parole chiave inserite dall'utente. & \hyperref[UC5point2]{UC5.2} \\
\hline
RF.OP.12 & L'utente deve poter visualizzare una lista delle parole chiave più frequenti. & \hyperref[UC5point3]{UC5.3} \\
\hline
RF.O.13 & L'utente deve poter visualizzare una singola parola chiave. & \hyperref[UC6]{UC6} \\
\hline
RF.O.14 & L'utente deve poter visualizzare i risultati dell'analisi di una parola chiave. & \hyperref[UC7]{UC7} \\
\hline
RF.O.15 & L'utente deve poter visualizzare la frequenza di una parola chiave. & \hyperref[UC7point1]{UC7.1} \\
\hline
RF.O.16 & L'utente deve poter visualizzare la densità di una parola chiave. & \hyperref[UC7point2]{UC7.2} \\
\hline
RF.D.17 & Il sistema deve mostrare un elenco dei tag in cui dovrebbe essere presente una parola chiave. & \hyperref[UC7point3]{UC7.3} \\
\hline
RF.D.18 & L'utente deve poter visualizzare un singolo tag. & \hyperref[UC7point3point1]{UC7.3.1} \\
\hline
RF.D.19 & L'utente deve poter visualizzare il nome del tag. & \hyperref[UC7point3point1point1]{UC7.3.1.1} \\
\hline
RF.D.20 & L'utente deve poter visualizzare il numero di occorrenze di una parola chiave nel tag. & \hyperref[UC7point3point1point2]{UC7.3.1.2} \\
\hline
RF.O.21 & Il sistema deve evidenziare tutte le occorrenze di una parola chiave nella pagina. & \hyperref[UC8]{UC8} \\
\hline
RF.O.22 & L'utente deve poter aggiornare l'analisi delle parole chiave. & \hyperref[UC9]{UC9} \\
\hline
RF.O.23 & Il sistema deve mostrare un avviso se il meta tag `keywords` non è presente. & \hyperref[UC10]{UC10} \\
\hline
RF.O.24 & Il sistema deve mostrare un avviso se la densità di una parola chiave è troppo alta. & \hyperref[UC11]{UC11} \\
\hline
RF.D.25 & Il sistema deve mostrare un avviso se il numero di occorrenze di una parola chiave nel tag è inferiore a una certa soglia. & \hyperref[UC12]{UC12} \\
\hline
RF.O.26 & Il sistema deve evidenziare le parole chiave con colori diversi in base al tag \gls{htmlg} che le racchiude. & Discussione con la Proponente \\
\end{longtable}

\newpage

\renewcommand{\arraystretch}{1.5}
\begin{longtable}{p{0.15\textwidth}p{0.55\textwidth}p{0.2\textwidth}}
\caption{Tabella dei requisti di qualità}
\label{tab:requisiti-qualitativi} \\
\hline\hline
\textbf{Requisito} & \textbf{Descrizione} & \textbf{Fonti}\\
\endfirsthead
    
\caption[]{Tabella dei requisiti di qualità (continua)} \\
\hline\hline
\textbf{Requisito} & \textbf{Descrizione} & \textbf{Fonti} \\ 
\endhead
    
\multicolumn{3}{r}{{Continua nella prossima pagina}} \\ 
\endfoot
    
\hline
\endlastfoot

\hline
RQ.O.1 & La funzionalità di analisi \gls{seog} deve rispettare le linee guida \gls{wcagg} 2.2, livello AA. & Discussione con la Proponente \\
\hline
RQ.OP.2 & L'estensione deve rimanere attiva su tutte le tab. & Discussione con la Proponente \\
\hline
RQ.D.3 & L'utente deve poter aprire e chiudere l'estensione in modo rapido e intuitivo. & Discussione con la Proponente \\
\end{longtable}

\renewcommand{\arraystretch}{1.5}
\begin{longtable}{p{0.15\textwidth}p{0.55\textwidth}p{0.2\textwidth}}
\caption{Tabella dei requisti di vincolo/dominio}
\label{tab:requisiti-vincolo} \\
\hline\hline
\textbf{Requisito} & \textbf{Descrizione} & \textbf{Fonti}\\
\endfirsthead
        
\caption[]{Tabella dei requisiti di vincolo/dominio (continua)} \\
\hline\hline
\textbf{Requisito} & \textbf{Descrizione} & \textbf{Fonti} \\ 
\endhead
        
\multicolumn{3}{r}{{Continua nella prossima pagina}} \\ 
\endfoot
        
\hline
\endlastfoot

\hline
RV.O.1 & La funzionalità di analisi \gls{seog} deve essere resa disponibile tramite \gls{github} (come codice sorgente) o pubblicata su un'altra piattaforma ad accesso pubblico. & Discussione con la Proponente \\
\hline
RV.O.2 & La funzionalità di analisi SEO deve essere integrata all'interno di un'estensione per Chrome & Discussione con la Proponente \\
\end{longtable}

\par\vspace{20pt}

\subsection{Riepilogo}

\begin{table}[H]
\centering
\caption{Tabella di riepilogo dei requisiti}
\label{tab:riepilogo-requisiti}
% MAX 12.5cm
\begin{tabular}{ccccc}
\hline\hline
\textbf{Requisito} & \textbf{Obbligatorio} & \textbf{Desiderabile} & \textbf{Opzionale} & \textbf{Totale} \\ 
\hline
Funzionale & 19 & 5 & 2 & 26 \\
\hline
Di qualità & 1 & 1 & 1 & 3 \\
\hline 
Di vincolo/dominio & 2 & 0 & 0 & 2 \\
\hline
\textbf{Totale} & 22 & 6 & 3 & \textbf{31} \\ 
\hline
\end{tabular}
\end{table}
