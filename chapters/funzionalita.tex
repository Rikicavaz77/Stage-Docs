\chapter{Funzionalità sviluppate}
\label{cap:funzionalità-sviluppate}

\section{Evidenziazione delle parole chiave}

\par La funzionalità “Highlight Keywords” consente di evidenziare tutte le occorrenze di una parola chiave all’interno della pagina web in cui l’estensione è attiva. Ogni occorrenza viene formattata con colori di sfondo, testo e bordo differenti, in base al tag \gls{html} che la contiene; tali colori sono personalizzabili tramite le impostazioni dell’estensione. Oltre alla personalizzazione cromatica delle occorrenze, ciascuna di esse è accompagnata da un’etichetta, posizionata in alto a sinistra, che riporta il nome del tag genitore. Nell’interfaccia grafica, a ogni keyword è associato un pulsante (highlighter) che permette di attivare o disattivare l’evidenziazione a livello globale. Questa funzionalità è accessibile sia dalla pagina principale sia da quella dedicata a una singola keyword, e lo stato del pulsante rimane sincronizzato tra le due. Le parole chiave inserite manualmente dall'utente possono essere evidenziate anche prima dell’avvio dell’analisi vera e propria, tramite il pulsante dedicato, mantenendo le funzionalità di analisi ed evidenziazione separate e indipendenti.

\subsection{Problematiche riscontrate}

\par La prima difficoltà è stata individuare un metodo efficiente per navigare il \gls{dom}, escludendo al contempo gli elementi non rilevanti dal punto di vista \gls{seo}. Tra le opzioni considerate, l’oggetto \textit{TreeWalker} si è dimostrato il più adatto, in quanto permette di gestire in modo centralizzato la logica di filtraggio dei nodi, determinando in modo preciso quali accettare o scartare. La sfida successiva ha riguardato la definizione del criterio di ricerca delle parole chiave. A differenza di strumenti come \textit{MozBar}, ho scelto di non evidenziare le occorrenze che compaiono come sottostringhe all’interno di parole più lunghe (ad esempio “Java” all’interno di “JavaScript”). In fase di ricerca, parole separate da un punto, un trattino (alto o basso) o un apostrofo vengono considerate un’unica parola. Tutti gli altri caratteri sono trattati come separatori, in modo analogo a quanto avviene nell’estensione \textit{SEOquake}.

\vspace{10pt}
\par\noindent Un altro ostacolo, particolarmente critico per l’affidabilità e le prestazioni del sistema, è emerso nel trattamento delle parole chiave composte da più termini separati da spazi (note anche come \textit{keyphrase}). L’approccio iniziale prevedeva l’esecuzione della ricerca sui singoli nodi di testo; di conseguenza, una keyword come “JavaScript Tutorial” non veniva riconosciuta se i due termini erano distribuiti su tag distinti (ad esempio, un h1 e uno span annidato). Questo limite è stato rilevato nel corso di test manuali condotti su un blog di intrattenimento e informazione, che adotta una convenzione particolare: evidenziare in grassetto solo il cognome all’interno dei nomi propri. Ciò comportava la frammentazione di molte \textit{keyphrase}, che non venivano identificate correttamente dal sistema. Per rendere la ricerca delle \textit{keyphrase} più granulare e accurata, ho sviluppato due algoritmi: uno ricorsivo e uno iterativo. Il primo suddivide la \textit{keyphrase} in blocchi e cerca ciascun blocco all’interno di nodi di testo adiacenti. Il secondo, invece, costruisce un testo virtuale concatenando il contenuto di nodi adiacenti, simulando il comportamento delle proprietà \textit{innerText} o \textit{textContent}. In entrambi gli approcci, i nodi adiacenti devono appartenere allo stesso contesto, ovvero condividere lo stesso elemento genitore ed essere elementi \textit{inline}. Tra le due soluzioni, è stato preferito il secondo algoritmo, in quanto più flessibile e maggiormente coerente con il comportamento reale del browser.

\section{Accessibilità}

\par Dal momento che la funzionalità di analisi delle parole chiave è integrata in un’estensione preesistente orientata all’accessibilità, anche questa componente è stata progettata in conformità alle linee guida \gls{wcag}. A tal fine, sono stati condotti test di accessibilità con i seguenti obiettivi:
\begin{itemize}
  \item Verificare il rispetto del \textbf{rapporto minimo di contrasto} tra testo e sfondo, pari a 4.5:1 per il testo di dimensioni “normali” e 3:1 per il testo di grandi dimensioni. Questo controllo è stato applicato sia agli elementi interni all’estensione, sia a quelli utilizzati per evidenziare le parole chiave;
  \item Controllare che le \textbf{immagini non decorative} siano dotate di un testo alternativo;
  \item Accertarsi che le \textbf{icone non decorative} siano accompagnate da un’etichetta accessibile;
  \item Verificare l’accessibilità dei \textbf{tooltip}, assicurandosi che vengano attivati e disattivati correttamente quando si interagisce con l’elemento trigger, sia tramite navigazione da tastiera sia tramite l’uso del mouse. Inoltre, i tooltip devono rimanere visibili al passaggio del mouse su di essi e devono poter essere chiusi premendo il tasto Esc;
  \item Controllare che tutti gli \textbf{elementi interattivi} (come pulsanti o link) dispongano di un’etichetta testuale descrittiva;
  \item Assicurare la \textbf{navigazione tramite tastiera}, con un ordine di tabulazione logico e una chiara indicazione visiva dell’elemento attualmente attivo.
\end{itemize}
